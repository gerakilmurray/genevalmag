\chapter{Multi-plans attribute grammar}
\label{chap:mag}
\minitoc

Tal como lo presenta Wuu-Yang en \cite{wuu-yang1} la familia de \textit{gramática de atributos multi-plans} es una clase que se encuentra dentro de las WDAG y dentro de ellas de las NC.

La familia de las MAG es estrictamente mayor que las ANCAG. La importancia de esta familia radica en que el procedimiento de computación de planes de evaluación estáticos toma \textbf{tiempo polinomial} en el numero de símbolos y producciones.

A continuación trataremos en detalle la familia de las MAG y en el capitulo siguiente abordaremos el mecanismo de evaluación de las mismas.


\section{Notación}
En esta sección, se define la notación que se usara en le desarrollo del presente capitulo para la definición de la familia MAG.
Básicamente, la notación usada es la utilizada por Wuu yang en \cite{wuu-yang1}, la cual proviene de la notación de kastens en \cite{kastens}.

\textbf{Definición} Una gramática de atributos esta dada por:

\begin{itemize}
\item Una CFG: (N, T, P, S) donde N es un conjunto finito de símbolos no terminales, T es un conjunto finito de símbolos terminales, S es un símbolo no terminal distinguido denominado \textit{símbolo inicial} y P es un conjunto de producción de la forma $X\rightarrow\alpha$ donde X es un símbolo no terminal ($X\in N$) y $\alpha$ es una cadena de terminales y no terminales. Una producción $\textit{p}\in P$ pude verse como
 $X_{0}\rightarrow\alpha_{0}X_{1}\alpha_{1}X_{2}\alpha_{2}\dots X_{k}\alpha_{k}$ 
donde \textit{$X_{0}, X_{1}, X_{2},\dots, X_{k}$} son símbolos no terminales y  \textit{$\alpha_{0}, \alpha_{1}, \alpha_{2}, \dots, \alpha_{k}$} son cadenas de símbolos terminales (posiblemente vacíos). Además se asume, que el símbolo inicial (S) no aparece en ninguna parte derecha de ninguna producción $\textit{p} \in P.$
\item Cada símbolo no terminal de la CFG tiene asociado un conjunto de \textit{atributos}. Cada uno de ellos describen las propiedades de una ``instancia'' especifica del simbol en el árbol sintáctico. Un atributo \textit{a} de un símbolo X se denota como \textit{X.a}. Una \textit{instacia} se define como la ocurrencia del símbolo, la cual esta dada por, además del símbolo y el atributo , también por un índice de aparición de ese símbolo en la producción y se denota como \textit{X[i].a} donde \textit{i} es el índice sintáctico de aparición del símbolo X. Cabe aclarar que se toma un \textit{i} para cada símbolo distinto en la producción. Por ejemplo, dada la producción $X\rightarrow Y X$ donde el conjunto de atributos se define como \textit{\{a,b\}} tanto para X como para Y, las posibles instancias son: \textit{X[0].a, X[0].b, X[1].a, X[1].a, Y[0].a} y \textit{Y[0].b}.
\end{itemize}

Un detalle de notación con respecto a CFG, en todos el desarrollo del trabajo se denotara a los símbolos no terminales con letras en mayúscula y a los símbolos terminales con letras en minúscula.

\section{Preliminares}

\subsection{\textit{DP} graph}
Los \textit{DP} graph denotan las relaciones de dependencias directas entre las instancias de la gramática. 

Informalmente, un \textit{DP} graph esta definido por un grafo (ver teoría de grafo) con las siguientes consideraciones: 

\begin{itemize}
\item Los nodos denotan \textit{instancias} de una producción.
\item Las aristas denotan la dependencia entre las instancias. Una arista $X[i].a\rightarrow Y[j].a$, \footnote{con \textit{i} y \textit{j} índices de ocurrencias consistentes para cada símbolo} denota que la evaluación de la instancia \textit{X[i].a} depende de la evaluación de \textit{Y[i].a} 
\end{itemize}

El \textit{DP} graph se denota como \textit{DP(p)} donde \textit{p} es una producción de la gramática. De manera mas formal, DP(p) se define como:

ME FALTA TEORÍA PARA ESTO.
% \begin{center}
%  $DP(p) = \{(X[i].a, Y[j].b) | X[i].a \rightarrow Y[j].b ∈ Rp \}$
% \end{center}


\subsection{\textit{Down} graph}

\subsection{\textit{DCG} graph}

\subsection{\textit{ADP} graph}



\section{Definición MAG}

Una gramática \textit{G} es una \textit{gramática de atributos multi-plans} si y solo si para toda producción \textit{q} en \textit{G}, todo grafo en \textit{SADP(q)} es no circular.
BLA BLA   
