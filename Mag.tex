\chapter{Gramática de Atributos Multi-planes}
\label{chap:mag}
\minitoc

Tal como lo presenta Wuu-Yang en \cite{wuu-yang1} la familia de \textit{gramática de atributos multi-planes} es una clase que se encuentra dentro de las WDAG y dentro de ellas de las NC.

La familia de las MAG es estrictamente mayor que las ANCAG. La importancia de esta familia radica en que el procedimiento de computación de planes de evaluación estáticos toma \textbf{tiempo polinomial} en el numero de símbolos y producciones.

A continuación trataremos en detalle la familia de las MAG y en el capitulo siguiente abordaremos el mecanismo de evaluación de las mismas.


\section{Gramática de atributos}
En esta sección, se define la notación que se usara en le desarrollo del presente capitulo para la definición de la familia MAG.
Básicamente, la notación usada es la utilizada por Wuu yang en \cite{wuu-yang1}, la cual proviene de la notación de Kastens en \cite{kastens}.

\begin{definition}
\label{def:grammarattr}
Una gramática de atributos es una tupla GA = (G, A, V, Dom, F, R) donde:
\begin{itemize}
\item G = (VN , VT , S, P ) es una gramática libre de contexto reducida y no ambigua.
\item A = $\cup_{X\in(VN \cup T)} A(X)$, es el conjunto finito de atributos (A(X) es el conjunto de atributos asociados al símbolo X)

\item V es el conjunto finito de dominios de valores de los atributos.
\item Dom : $A\rightarrow V$ asocia a cada atributo un dominio o conjunto de valores d ∈ V .
\item F es un conjunto finito de funciones semánticas de la forma:
\begin{equation}
f \subseteq (\otimes_{j=0}^{k}{ Dom(a_{j} ))\rightarrow Dom(a_{0})}
\end{equation}

\item R = $\bigcup _{p∈P} Rp$ es el conjunto finito de reglas de atribución o ecuaciones asociadas a cada producción p ∈ P , donde
\begin{equation}
R^{p} = \bigcup\limits_{j=0}^{m^{p}}{\{R_{j}^{p}\}}\ \ \ \ \ \ (\#(R^{p} ) = m^{p} ≥ 0)
\end{equation}
y cada regla $r_{j}^{p} \in R^{p}$ , con 0 ≤ j ≤ mp es una tupla de la forma

\begin{equation}
r_{j}^{p} = f(X_{0}.a0 ,\dots , X_{k}.ak)
\end{equation} 
con $X_{i} \in X^{p}$ , $a_{i} \in A(X_{i})$, (0 ≤ i ≤ k) y $f \in F$.

\end{itemize}
\end{definition}

Detalles de notación:
\begin{itemize}
\item Con respecto a CFG, en todos el desarrollo del trabajo se denotara a los símbolos no terminales con letras en mayúscula y a los símbolos terminales con letras en minúscula.
\item Se utilizará la notación \textbf{X.a} para significar que el atributo \texttt{a} está asociado al símbolo \texttt{X} (a $\in$ A(X)) y para denotar el valor de una ocurrencia o instancia del attributo \texttt{a} del símbolo \texttt{X} en una regla de atribución.

\end{itemize}
En \cite{tesismarcelo} (capito 2) podemos encontrar detalles sobre las definiciones arriba analizadas, como asi tambien, conceptos que fueron considerados en el desarrollo del proyecto.  

\section{Preliminares}

\subsection{Grafo \textit{DP}}
Los grafos \textit{DP} denotan las relaciones de dependencias directas entre las instancias de la gramática. 

Un grafo \textit{DP} esta definido (ver teoría de grafo) con las siguientes consideraciones: 

\begin{itemize}
\item Los nodos denotan \textit{instancias} de una producción.
\item Las aristas denotan la dependencia entre las instancias. Una arista $X_{i}.a\rightarrow X{j}.b$, \footnote{con \textit{i} y \textit{j} índices de ocurrencias consistentes con alguna ecuacion de la gramatica} denota que la evaluación de la instancia \textit{$X_{i}.a$} depende de la evaluación de \textit{$Y_{i}.b$} 
\end{itemize}

El conjunto de dependencias directas de una producción, de la gramtica, se denota como \textit{DP(p)}(\textit{p} produccion de la gramática) y se define como:
 \begin{definition}
Dada una produccion \textit{p} de una gramatica de atributos definida como en \ref{def:grammarattr}, entonces
\begin{equation}
 DP(p) = \{(X_{i}.a, X_{j}.b) | X_{i}.a \rightarrow Y_{j}.b \in R^{p} \}
\end{equation}
\end{definition}

\subsection{Grafo \textit{Down}}
Los grafos \textit{Down} denotan las relaciones de los atributos de un simbolo. 

Un grafo \textit{Down} esta definido (ver teoría de grafo) con las siguientes consideraciones: 

\begin{itemize}
\item Los nodos denotan \textit{atributos} de un simbolo.
\item Las aristas denotan la dependencia entre los atributos de un simbolo. Dado los atributos $a$ y $b\in A(X)$ del simbolo $X\in VN$, una arista $a\rightarrow b$, denota que la evaluación del atributo \texttt{a} depende de la evaluación de \texttt{b}. 
\end{itemize}
El conjunto de dependencias entre los atributos de un simbolo se denota como  
\texttt{Down(X)}($X\in VN$ de una gramtica G) y se define:
\begin{definition}
Dada un simbolo \textit{X} de una gramatica de atributos definida como en \ref{def:grammarattr}, entonces
\begin{equation}
 Down(X) = \{(a,b) | a \rightarrow b \} con a,b \in A(X)
\end{equation}
\end{definition}
\subsection{Grafo \textit{DCG}}

\textit{DCG} significa \textit{downward characteristic graphs}, los mismos contienen las dependencias entre instancias de la gramatica para una produccion \textit{p}, teniendo en cuenta un simbolo en la gramatica.
\begin{definition}
Dado \textit{q} una produccion de la forma $X_{0}\rightarrow \alpha_{0} X_{1} \alpha_{1} X_{2} \dots X_{k} \alpha_{k}$, el \textit{downward characteristic graph} of $X_{0}$ en los subarboles derivados via la produccion \textit{q}, denotado como $DCG_{X_{0}}(q)$, es un grafo donde: 
\begin{itemize}
\item Los nodos son atributos del simbolo $X_{0}$.
\item Una arista, $X.a \rightarrow X.b$, denota una dependencia (transitiva) de X.b sobre X.a en algun subarbol derivado desde $X_{0}$ via \textit{q}.
\end{itemize}
\end{definition}
Tomemos el siguiente teorema presentado por Wuu-Yang en \cite{wuu-yang1}:
\begin{theorem}
 $\bigcup\limits_{\textit{todo p}}{}{DCG_{X} (p) = Down (X)}$
\end{theorem}
\underline{Nota:} $DCG_{X}(p)$ contiene las depndencias, entre las instancias de la gramatica, para el simbolo \texttt{X}, acotando el analisis para la produccion \textit{p} y los posibles contextos inferiores.


En la seccion XXX se analiza el algoritmo para contruir los grafos DCG.

\subsection{Grafo \textit{ADP}}

Las siglas \textit{ADP} significan \textit{augmented dependency graph}. El grafo \textit{ADP} esta definido por instancias de la gramatica en los nodos y una arista estan definidas como: $X_{i}.a\rightarrow X{j}.b$, denota que la evaluación de la instancia \textit{$X_{i}.a$} depende de la evaluación de \textit{$Y_{i}.b$}

El conjunto de dependencias se denota como $ADP (q | p_{1}, p_{2}, \dots, p_{k})$ y se define:
\begin{definition}
\begin{equation}
 ADP (q | p_{1}, p_{2}, \dots, p_{k}) = DP(q) \bigcup\limits_{k}^{i=1}{DGC_{X_{i}}} (p_{i})
\end{equation}
\end{definition}


\section{Definición MAG}

Una gramática \textit{G} es una \textit{gramática de atributos multi-planes} si y solo si para toda producción \textit{q} en \textit{G}, todo grafo en \textit{SADP(q)} es no circular.
BLA BLA   
