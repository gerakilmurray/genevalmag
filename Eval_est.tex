\chapter{Evaluación estática de MAG}
\label{chap:eval_est}
\minitoc

En este capítulo se presentan los algoritmos de construcción de cada tipo de grafo, así como los usados para el cómputo de planes de evaluación y sus respectivas secuencias de visita. Los mismos se basan en los propuestos por Wuu Yang en \cite{wuu-yang1}. Mientras que el algoritmo para el cálculo de secuencias de visita fue íntegramente desarrollado e implementado en el marco de la tesis.

Para comprobar la correctitud de los algoritmos y lograr mayor grado de eficiencia, los mismos fueron testeados mediante la herramienta \textit{gcov}\footnote{Detalles en sección \ref{sec:metoherram}.}. La cual fue utilizada con mayor énfasis para el algoritmo de cómputo de secuencias de visita. En conjunto con los resultados proporcionados por \textit{gcov} y análisis teóricos, la heurística que implementa el algoritmo fue refinada en varias oportunidades.

\section{Algoritmos para cómputo de grafos}

Todos estos algoritmos tienen una característica en común, expresan las dependencias existentes entre los atributos de una gramática desde distinto aspectos. Por lo tanto en su construcción, los nodos y aristas son obtenidos de las relaciones entre dichos elementos.

\subsection{Cómputo de DP}
\label{subsec:alg-DP}
Se debe generar un grafo DP por cada regla de la gramática, representando las dependencias directas entre los atributos de los símbolos que participan en la producción, tal como se analizó en el capítulo anterior en la sección \ref{subsec:graph-dp-def}.

Es el más directo de construir, ya que por cada ecuación de la regla actual, se agrega una arista entre cada uno de los símbolos de \textbtt{r\_value} con destino al \textbtt{l\_value} de la ecuación. Obviamente se deben agregar los nodos necesarios y sin repeticiones, para cada una de las instancias de atributos. Ver algoritmo \ref{alg:graphdp}.

\begin{algorithm}[!ht]
\lstinputlisting[numbers=left, columns=fullflexible]{input_file_code/compute_dp_wuu_yang}
\vspace{-0.5cm}
\caption{\label{alg:graphdp}Cómputo de grafos \textit{DP}.}
\end{algorithm}

\subsection{Cómputo de Down}
\label{subsec:alg-DOWN}
Este tipo de grafo, se debe crear para cada uno de los símbolos no terminales de la gramática. Representan las relaciones de dependencia entre solamente sus atributos, considerando todas las reglas en las que participan (Ver sección \ref{subsec:graph-down-def} del capítulo anterior). Para más detalle, ver algoritmo \ref{alg:graphdown}.

\begin{algorithm}[!ht]
\lstinputlisting[numbers=left, columns=fullflexible]{input_file_code/compute_down_wuu_yang}
\vspace{-0.5cm}
\caption{\label{alg:graphdown}Cómputo de grafos \textit{Down}.}
\end{algorithm}

La función \textbtt{project} recibe un grafo, del cual debe eliminar todos los nodos y aristas relacionadas que no pertenecen al conjunto de atributos que le pasan como segundo parámetro. Pero para evitar que al eliminar nodos y aristas el grafo pierda información sobre las dependencias entre los atributos, se calcula previamente la clausura transitiva del grafo. La implementación de \textbtt{project} se presenta en el algoritmo \ref{alg:graphproj}.

\begin{algorithm}[!ht]
\lstinputlisting[numbers=left, columns=fullflexible]{input_file_code/project_wuu_yang}
\vspace{-0.5cm}
\caption{\label{alg:graphproj}Proyección sobre grafos.}
\end{algorithm}

\subsection{Cómputo de DCG}
\label{subsec:alg-DCG}
Estos grafos se generan para cada regla de la gramática y considerando las dependencias entre los atributos del símbolo de la parte izquierda de la producción (\textbtt{head}). Para computarlo, tal como se lo definió en la sección \ref{subsec:graph-dcg-def}, se deben unir el grafo \textit{DP} de la regla junto a todos los grafos \textit{Down} de los símbolos de la parte derecha de la regla (\textbtt{body}) y luego se utiliza, nuevamente, la función \textbtt{project}. Para más detalle, ver algoritmo \ref{alg:graphdcg}.

\begin{algorithm}[!ht]
\lstinputlisting[numbers=left, columns=fullflexible]{input_file_code/compute_dcg_wuu_yang}
\vspace{-0.5cm}
\caption{\label{alg:graphdcg}Cómputo de grafos \textit{DCG}.}
\end{algorithm}

\subsection{Cómputo de ADP}
\label{subsec:alg-ADP} 
Para generar estos grafos, Wuu Yang no plantea un algoritmo preciso aunque no es necesario ya que toda la heurística necesaria se encuentra en la definición de los mismos (ver \ref{mag:adpdef}).

Un grafo ADP corresponde a una regla bajo una cierta combinación de producciones para cada uno de sus hijos. Es decir, que para una misma regla pueden existir varios grafos ADP, ya que sus hijos pueden producir más de un contexto. El cómputo de grafos ADP es presentado en el algoritmo \ref{alg:graphadp}.

\begin{algorithm}[!ht]
\lstinputlisting[numbers=left, columns=fullflexible]{input_file_code/compute_adp_wuu_yang}
\vspace{-0.5cm}
\caption{\label{alg:graphadp}Cómputo de grafos \textit{ADP}.}
\end{algorithm}

\section{Algoritmo de cómputo de planes}
\label{sec:comp-planes}
Luego de verificar que la gramática es MAG\footnote{Ver \ref{def:MAG}.}, se deben generar los planes de evaluación para la gramática. Cada ADP producirá un plan de evaluación, por lo que ambos tendrán el mismo identificador, es decir, la regla a la cual pertenecen y el contexto que se aplicó a los símbolos no terminales de la misma.

Este algoritmo propuesto por Wuu Yang, es detallado en profundidad en esta sección y la correctitud del algoritmo es abordado por Wuu Yang en \cite{wuu-yang1}.

La idea principal es utilizar una cola de trabajos, en la cual se insertan pares que contienen una regla y un orden de evaluación para sus instancias. Para cada uno de los grafos ADP asociados a la regla, se le computa su orden de evaluación para el contexto especificado por el grafo.

Luego, para cada uno de los símbolos no terminales de la parte derecha de la regla, considerando la producción que le corresponde según el contexto, se obtiene el plan proyectado. Ese nuevo par, se intenta insertar en la cola de trabajos para seguir con el cómputo de planes (Ver algoritmo \ref{alg:planeval}).

\begin{algorithm}[!ht]
\lstinputlisting[numbers=left, columns=fullflexible]{input_file_code/compute_plan_wuu_yang}
\vspace{-0.5cm}
\caption{\label{alg:planeval}Cómputo de planes de evaluación.}
\end{algorithm}

La función \textbtt{compute\_order}, debe agregarle al grafo ADP la información inducida por el orden para evaluar los atributos del símbolo \textbf{head} de la regla a la que pertenece el grafo.

Para ello, como todo orden se puede ver como una secuencia, se agregan al grafo un nodo por cada elementos del orden y las aristas para todos los pares posibles\footnote{Ver más detalles en \ref{sec:genplanes}} (Ver algoritmo \ref{alg:compord}).

\begin{algorithm}[!ht]
\lstinputlisting[numbers=left, columns=fullflexible]{input_file_code/compute_order_wuu_yang}
\vspace{-0.5cm}
\caption{\label{alg:compord}Cómputo de orden de evaluación.}
\end{algorithm}

La funcionalidad que cumple \textbtt{project} (en el cómputo de planes), es dado un orden de evaluación purgarlo de todo elemento ajeno a el conjunto de ocurrencias de atributos especificados como parámetro. Previamente se debe aplicarle la clausura transitiva para conservar las relaciones de precedencia entre las ocurrencias (Ver algoritmo \ref{alg:projordeval}).

\begin{algorithm}[!ht]
\lstinputlisting[numbers=left, columns=fullflexible]{input_file_code/project2_wuu_yang}
\vspace{-0.5cm}
\caption{\label{alg:projordeval}Proyección sobre orden de evaluación.}
\end{algorithm}

\subsection*{Resultados}

\begin{definition}
Sea $\Psi$ un orden (parcial o total) sobre un conjunto de atributo y sea $\omega$ un orden (parcial o total) sobre un subconjunto de $\Psi$. Se dice que $\Psi$ es compatible con $\omega$ si para todo a $\rightarrow$ b $\in$ $\omega$ es una arista en $\Psi$.
\end{definition}


\begin{theorem}
\label{th1}
Sea q la producción X$_{0}$ $\rightarrow$ $\alpha_{0}$ X $_{1}$ $\alpha_{1}$ X $_{2}$ $\alpha_{2}$ $\ldots$ X $_{k}$ $\alpha_{k}$. Dentro de \textbtt{compute\_order}($ADP (q | p_{1}, p_{2}, \dots, p_{k})$, $\omega$), del algoritmo \ref{alg:compord}, si el $ADP (q | p_{1}, p_{2}, \dots, p_{k})$ no contiene ciclos y $\omega$ es compatible con $DCG_{X_{0}}(q)$, entonces

\begin{enumerate}
    \item \label{th1:p1}La introducción de las aristas \{a$_{i}$ $\rightarrow$ a$_{i+1}$ | i = 1, $\ldots$ , m-1\} dentro de la función \textbtt{compute\_order} no introduce ciclos dentro del $ADP (q | p_{1}, p_{2}, \dots, p_{k})$
 
    \item \label{th1:p2}El orden total resultante $\Psi$ es compatible con $DCG_{X_{i}}(p_{i})$, para cada símbolo X$_{i}$ no terminal del \textbtt{body} de la producción q, que es, a su vez \textbtt{head} de la producción p$_{i}$.
\end{enumerate}

\end{theorem}

\begin{proof}[\textbf{Demostración}]

Notar que en \textbtt{compute\_order}($ADP (q | p_{1}, p_{2}, \dots, p_{k})$, $\omega$), el orden total $\omega$ es representado por el conjunto de aristas \{a$_{i}$ $\rightarrow$ a$_{i+1}$ | i = 1, $\ldots$ , m-1\}, donde \{a$_{i}$ | i = 1, $\ldots$ , m\} son las ocurrencias de atributos de X$_{0}$. Si la intrucción de aristas \{a$_{i}$ $\rightarrow$ a$_{i+1}$ | i = 1, $\ldots$ , m-1\} introduce un ciclo dentro de $ADP (q | p_{1}, p_{2}, \dots, p_{k})$, $\omega$ es incompatible con la proyección de $ADP (q | p_{1}, p_{2}, \dots, p_{k})$ sobre las ocurrencias de atributos de X$_{0}$, que es un subgrafo $DCG_{X_{0}}(q)$. Por lo tanto, $\omega$ debe ser incompatible con $DCG_{X_{0}}(q)$, lo cual contradice lo supuesto en el teorema. Esto prueba el punto \ref{th1:p1} del teorema.

El orden total $\Psi$ computado por la función \textbtt{compute\_order} es compatible con $ADP (q | p_{1}, p_{2}, \dots, p_{k})$. Notar que el grafo $DCG_{X_{i}}(p_{i})$, de cada símbolo no terminal X$_{i}$ de la parte derecha, es un subgrafo de $ADP (q | p_{1}, p_{2}, \dots, p_{k})$. Por lo tanto, $\Psi$ es compatible con $DCG_{X_{i}}(p_{i})$, para cada símbolo no terminal X$_{i}$ del \textbtt{body}, que es, a su vez \textbtt{head} de la producción p$_{i}$.

\end{proof}

\begin{theorem}
\label{th2}
El algoritmo \textbtt{compute\_plans} (ver \ref{alg:planeval}) computa correctamente un conjunto de ordenes de evaluación para todas las producciones alcanzables.
\end{theorem}

\begin{proof}[\textbf{Demostración}]
Es suficiente probar los siguientes puntos:
\begin{enumerate}
    \item \label{th:p1}Todo $ADP (q | p_{1}, p_{2}, \dots, p_{k})$ no contiene ciclos.

    \item \label{th:p2}Para todo \textbtt{compute\_order}($ADP (q | p_{1}, p_{2}, \dots, p_{k}), \omega)$ se cumple que $\omega$ es compatible con $DCG_{X_{0}}(q)$,  donde $X_{0}$ es el \textbtt{head} de la producción $q$.
\end{enumerate}

El punto \ref{th:p1} es consecuencia directa de la definición de gramática MAG. Para el punto \ref{th:p2} se probará el siguiente invariante:
$$
Si\ (q,\omega)\in WL\ entonces\ \omega\ es\ compatible\ con\ DCG_{X_{0}}(q)
$$

El algoritmo \textbtt{compute\_plans} es un método iterativo, donde el primer $\omega$ es un orden $\mu$ arbitrario para los atributos del símbolo inicial $S$. Dado que no existe dependencias en los atributos del símbolo inicial, $\mu$ es trivialmente compatible con $DCG_{S}(q)$. Las iteraciones posteriores del algortimo \textbtt{compute\_plans} se aplican a ordenes $\omega$ que son proyeccciones de $\Psi$ para atributos de los símbolos no terminales $X_{i}$ del \textbtt{body} de la producción $p$, donde $p$ $\in$ al contexto inferior de la producción $q$ (teniendo en cuenta los posibles árboles sintácticos). Notar que $X_{i}$ es símbolo no terminal de la parte derecha de la producción $p$ y el símbolo \textbtt{head} de la producción $q$ y $\Psi$ es un orden de evaluación para las ocurrencias de los atributos en la producción $p$. Por el teorema \ref{th1} se cumple que los ordenes de las proyecciones de $\Psi$ son compatibles con $DCG_{X_{i}}(q)$.

De los puntos \ref{th:p1} y \ref{th:p2}, todos los ordenes de evaluación $\Psi$ computados por \textbtt{compute\_order} son compatibles con los ordenes parciales para $ADP (q | p_{1}, p_{2}, \dots, p_{k})$ $\cup$ $\omega$, por lo tanto, la evaluación es consistente para la producción $q$ cuando los atributos del símbolo \textbtt{head} de $q$ son evaluados en el orden $\omega$ y $p_{1}, p_{2}, \dots, p_{k}$ son producciones aplicadas a los símbolos no terminales del \textbtt{body} de $q$. Como el algoritmo es un método exaustivo, podemos afirmar que todas las producciones alcanzables $q$ y todos los ordenes $\omega$ necesarios son examinados por el algoritmo.

\end{proof}

% Un resultado muy importante a tener en cuenta es el echo de que el algoritmo \ref{alg:planeval} genera la minima cantidad de planes que son necesarios para la evaluacion de una gramatica.
% 
% Este resultado es expresado mediante un teorema (teorema \ref{teo:planes_plausibles}), pero antes de llegar al mismo es recomendable tener en cuenta algunos aspectos y definiciones.
% 
% En primer lugar, en todo grafo ADP se almacenan las dependencias entre instancias de atributos, segun; las \textit{dependencias directas}, las dependencias impuestas por el \textit{contexto superior} y las del \textit{contexto inferior}. Entonces, instancias diferentes de una producción $p$ tendrán las mismas dependencias directas (Grafo DP), pero podrán tener diferentes dependencias impuestas por los contextos inferiores y superiores. Estas, y demás observaciones deben tenerse en cuenta para la construcción de un plan de evaluacion. En consecuencia, es necesario determinar todas las \emph{dependencias posibles} entre las instancias de los atributos que ocurren en una producción para luego poder determinar todas las \emph{dependencias plausibles} entre los atributos que ocurren en una producción. 
% 
% \begin{definition} Una partición posible de un grafo de dependencias $G\_T=<V\_N,A>$ es una partición sobre $V\_N$ tal que dados dos instancias de atributos $X.a$ y $Y.b$, si $(X.a,Y.b) \in A$, entonces $[X.a] = [X.b]$ (están en la misma clase).
% \end{definition}
% 
% \begin{definition} Sea $\pi$ una partición posible de un grafo de dependencias $G_T=<V_N,A>$. Sea $V_{N_X}$ un nodo rotulado $X$ en el árbol sintáctico $T$. La proyección de $\pi$ en las instancias de los atributos de $X$ es una partición posible de los atributos de $X$.
% \end{definition}
% 
% Una \textit{partición plausible} es un refinamiento de una partición posible, dado que las restricciones impuestas por algún contexto superior puede que dos clases de una partición plausible se unan en una partición posible (en el caso de una dependencia transitiva entre dos instancias de atributos impuesta por el contexto superior). Para el símbolo de comienzo la partición plausible coincide con una partición posible (ya que no tiene contexto superior). \\
% 
% 
% \begin{definition}
% \label{def:plan_espurios}
% Dada una MAG, se define \textbf{plan espurio} a aquel plan de evaluación para el cual no es posible encontrar un AST para dicho plan.
% 
% NOTA: podriamos ver como: proviene de una particion posible que no puede ser plausible.
% \end{definition}
% 
% \begin{definition}
% \label{def:plan_plausible}
% Dada una MAG, se define \textbf{plan plausible} a aquel plan de evaluación para el cual existe al menos un AST para dicho plan.
% 
% NOTA: lo podriamos ver como: un plan es plausible si proviene de la union de particiones plausibles.
% \end{definition}


%\subsection{Teorema1}
%\label{teo:planes_plausibles}
%El algoritmo \ref{alg:planeval} genera planes consistentes con los ADP construidos.

% El algoritmo \ref{alg:planeval} genera la minima cantidad de planes o dicho en otras palabras, genera solamente planes plausibles.

%\underline{\emph{Demostración:}}

% SACAR ESTE Y EL ANTERIOR DESDE EL PAPER DE WUU YANG DE CLASIFICACION.
%\subsection{Teorema2}
%\label{teo:ciclo_compute}
%El algoritmo \ref{alg:compord} no introduce ciclos en los planes.

%\subsection{Teorema3}
%\label{teo:reglas_alcan}
%El algoritmo \ref{alg:planeval} genera planes para todas las producciones alcanzables de la gramatica y tiene en cuenta todos los ordenes $\omega$ necesarios.

\section{Algoritmo de cómputo de secuencias de visita}
\label{sec:algseqvisit}

Las secuencias de visitas, son traducciones de los planes de evaluación en secuencias compuestas por tres comandos\footnote{\textbtt{compute}, \textbtt{leave} y \textbtt{visit}, ver más detalles en \ref{sec:sec-visit}.} que permiten la navegación por el AST para su evaluación.

Este proceso fue diseñado e implementado para el desarrollo de \maggen. La generación se dividió en dos algoritmos, uno encargado de la inicialización de todas las secuencias de visitas y la invocación del otro, solamente, para los planes de evaluación que pertenecen a la producción inicial de la gramática. Este último algoritmo calcula las secuencias de visitas totales.

La traducción bajo demanda de los planes de evaluación, permiten realizar optimizaciones como se explicará en el capítulo \ref{chap:implem} de implementación. El cómputo de secuencias de visita es presentado en el algoritmo \ref{alg:genvisitseq}.

\begin{algorithm}[!ht]
\lstinputlisting[numbers=left, columns=fullflexible]{input_file_code/generate_visit_sequences}
\vspace{-0.5cm}
\caption{\label{alg:genvisitseq}Generador de secuencias de visitas.}
\end{algorithm}

El algoritmo que realiza el cómputo de las secuencias de visita totales realiza un proceso recursivo para lograr la traducción. Se lo puede considerar como un recorrido sobre los posibles contextos\footnote{En la sección \ref{sec:constseqvisit} se ampliará esta heurística.}. Por lo que se debe llevar una secuencia de visita local, a la cual se le insertarán los distintos comandos. Además, será necesario una función que lleve las instancias de atributos ya calculadas.

De esta forma, cuando se detecta un \textbtt{visit}, se debe localizar a que hijo tiene que recursar en base al plan actual, el cual tiene un contexto establecido. Además se debe determinar el plan proyectado que actuará de orden de evaluación impuesto en la recursión. Al volver se debe actualizar todas las instancias de atributos que fueron computadas en el contexto inferior, para evitar futuras visitas innecesarias.

Con el mismo criterio, al producirse un \textbtt{leave} por la necesidad de requerir que en el contexto superior se calcule una instancia de atributo, puede ser que se calculen más de lo que solicitó, por eso se deben actualizar la información de computados.

Una ecuación se podrá computar solamente cuando todo su \textbtt{r\_value} halla sido marcado como computado, ya sea mediante un \textbtt{visit} o \textbtt{leave}.

Por último, es necesario una función que combine y actualice la secuencia de visita global, con la que se calculó a nivel local. Esto es, ya que, una secuencia de visita puede ser completada en más de una visita. Todo el proceso analizado anteriormente es presentado en el algoritmo \ref{alg:genvisitseqrec}.

\begin{algorithm}[!ht]
\lstinputlisting[numbers=left, columns=fullflexible]{input_file_code/gen_visit_seq}
\vspace{-0.5cm}
\caption{\label{alg:genvisitseqrec}Función recursiva de generación de secuencias de visita.}
\end{algorithm}

\section{Algoritmo de evaluación de atributos}
\label{sec:algevalattr}

Este algoritmo, no se encuentra implementado dentro de \maggen, sino que dentro del código del evaluador generado.

Se basa en dos etapas bien definidas y con propósitos diferentes. En las cuales se deben realizar dos recorridos del AST de entrada.

El primero es una barrida descendente en la que se seleccionan los planes de evaluación para cada nodo, para lo cual se debe armar su contexto en base a la regla referenciada por el nodo actual y la que poseen cada uno de sus hijos. El orden de evaluación de atributos para la raíz del AST es aleatoria, ya que sólo contiene símbolos sintetizados y no pueden ser utilizados en otro contexto por ser una gramática extendida.

En el segundo recorrido, no precisamente se realiza en un sentido uniforme, sino que se navega descendente y ascendentemente, de acuerdo a la secuencia de visita correspondiente al plan seleccionado en la primera etapa. 

La primera etapa es realizada en la funcion \textbtt{traverse}. Esta, tiene toda la información necesaria ya precargada, gracias al procesamiento previo de los anteriores algoritmos. Debe determinar el plan de evaluación, almacenado en $\Gamma$, armando la clave con la regla asociada al nodo corriente y las reglas asociadas de todos los hijos. El orden de evaluación de atributos $\omega$, le viene como parámetro, el cual corresponde con un plan de evaluación proyectado, que desde el contexto superior le impusieron. Una vez seleccionado el plan, también queda establecida la secuencia de visita, para el nodo.

Para continuar con el proceso, se debe seleccionar para cada hijo el plan proyectado desde $\Theta$, que será el orden que le imponen para la evaluación de sus atributos. Ver algoritmo \ref{alg:traverse}.

\begin{algorithm}[!ht]
\lstinputlisting[numbers=left, columns=fullflexible]{input_file_code/traverse_wuu_yang}
\vspace{-0.5cm}
\caption{\label{alg:traverse}Función Traverse.}
\end{algorithm}

La segunda etapa es realizada por la función \textbtt{eval\_visiter}. Esta implementa la evaluación guiada por la secuencia de visita. Este algoritmo, Wuu Yang no lo presenta y lo deja a libre elección del usuario. En \ref{sec:codcppalgeval} se presenta el evaluador guiado por visita implementado para \maggen. 

El algoritmo teórico que escogido es el mostrado en \ref{alg:evalvisiter}.

\begin{algorithm}[!ht]
\lstinputlisting[numbers=left, columns=fullflexible]{input_file_code/eval_visiter_wuu_yang}
\vspace{-0.5cm}
\caption{\label{alg:evalvisiter}Evaluador guiado por secuencia de visita.}
\end{algorithm}

Las dos etapas anteriores son realizadas en el algoritmo \ref{alg:evalattr} denominado \textbtt{eval}. 

\begin{algorithm}[!ht]
\lstinputlisting[numbers=left, columns=fullflexible]{input_file_code/eval_attribute_wuu_yang}
\vspace{-0.5cm}
\caption{\label{alg:evalattr}Evaluación de atributos.}
\end{algorithm}