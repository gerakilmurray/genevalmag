\chapter{Clasificación de gramática de atributos}
\label{chap: clas_ag}
\minitoc

La clasificación de las gramática de atributos puede verse teniendo en cuenta dos aspectos: según la\textbf{ estrategia de evaluación} y según las \textbf{dependencias entre los atributos}. 

Cada una de las familias de gramática de atributos tiene su propio poder expresivo, como así también, sus restricciones en el método de evaluación. La relación entre estos dos aspectos esta dado por el siguiente razonamiento: \textit{mientras mas restricciones en el método de evaluación, menos poder expresivo}.

En el capitulo siguiente trataremos la familia de las gramática de atributos multi-planes (MAG) que son las usadas en el marco de este trabajo. A continuación trataremos otras familias que contribuyen a la teoría de gramática de atributos.

\section{Clasificación basada en la estrategia de evaluación}

hablar de las WDAG y también de las OAG. Tener en cuenta (Marcelo p 44)

        
AG      Gramáticas de atributos
WDAG    Gramáticas de atributos bien definidas   
ANCAG   GA absolutamente no circulares
EOAG    GA Ordenadas Extendidas
OAG     GA Ordenadas       
m-APAG  GA evaluables en m pasadas alternantes       
n-PAG   GA evaluables en n pasada       
L-AG    GA l-atribuida        
S-AG    GA s-atribuidas


\section{Clasificación basada en dependencias}

bla bla
\section{Clasificación de Knuth}

cla cla
\subsection{Árbol sintáctico atribuido}


\subsection{Gramáticas no circulares(NC)}
bla bla

\subsection{ANCAG}
