\bibliographystyle{unsrt}
\begin{thebibliography}{10}

\bibitem {wuu-yang1} Wuu-Yang. 1998. \textit{Multi-Plan Attribute Grammars}. Department of Computer Information Science. National Chiao-Tung University, Hsin-Chu, Taiwan, R.O.C.

\bibitem {wuu-yang2} Wuu-Yang. 1999. \textit{A Classification of Non Circular Attribute Grammars based on Lookahead behavior}. Department of Computer and Information Science. National Chiao-Tung University, Hsin-Chu, Taiwan, R.O.C.

\bibitem {wuu-yang3} Wuu-Yang. 1998. \textit{Conditional Evaluation in Simple Multi-Visit Attribute Grammar Evaluators}. Department of Computer and Information Science. National
Chiao-Tung University, Hsin-Chu, Taiwan, R.O.C.

\bibitem {gramatica} John E. Hopcroft, Rajeev Montwani, Jefrey D. Ullman.\textit{Introduction to Automata theory, languajes, and computation}. Addison-Wesley (2001) second edition.

\bibitem {compiladores} Alfred V. Aho, Ravi Sethi, Jeffrey D. ullman.\textit{Compilers: Principles, Techniques and Tools}. Addison-Wesley (1985)  Iberoamericana, S.A. Wilmington, Delaware, E.U.A.

\bibitem {tesismarcelo} Arroyo, Marcelo Daniel. \textit{Gramáticas de atributos, clasificación y aportes en técnicas de evaluación}. Tesis de carrera de Magister en Ciencias de la Computación. Universidad Nacional del Sur. Bahía Blanca - Argentina.

\bibitem {kastens} U. Kastens. 1980. \textit{Ordered Attribute Grammars}. Acta Informatica. Vol. 13, pp. 229-256.

\bibitem {meyer} Bertrand, Meyer (1997). \textit{Object-Oriented Software Construction}. Segunda edición. Prentice Hall Professional Technical Reference. Santa Barbara (California).

\bibitem {estruc-algorit} Alfred V. Aho, John E. Hopcroft, Jefrey D. Ullman. \textit{Data Structures and Algorithms}. Addison-Wesley publishing Company, Reading, Massachusetts, E. U. A.(1983).

\bibitem {valen} Valentin David. \textit{Attribute Grammars for C++ Disambiguation}. LRDE, 2004.

\bibitem {Knuth} D. Knuth. 1968. \textit{Semantics of context free languages}. Math Systems Theory 2.June 2.Pag: 127-145.


% \bibitem {svn} \textbf{subversion}. \textit{Enterprise-class centralized version control for the masses.} URL:\texttt{http://subversion.apache.org/}.
% 
% \bibitem {eclipse} \textbf{eclipse}. \textit{Multi-language software development environment.} URL:\texttt{http://www.eclipse.org/}.
% 
% \bibitem {doxy} \textbf{doxygen}. \textit{A documentation generator for C++, C, Java, Objective-C, Python, IDL (CORBA and Microsoft flavors), Fortran, VHDL, PHP and C\#}. URL: \texttt{www.doxygen.org}.

\bibitem{c++1} \textbf{C++}. \textit{C++ Annotations Version 8.2.0.} URL: \texttt{http://www.icce.rug.nl/documents/cplusplus/}. 

\bibitem{c++2} \textbf{C++}. \textit{C++ reference} URL: \texttt{http://www.cppreference.com/wiki/es/start}. 

\bibitem{latex} \textbf{\LaTeX}. \textit{A document preparation system} URL: \texttt{http://www.latex-project.org/}.

\bibitem{boost} \textbf{Boost}. \textit{Boost C++ Libraries}.\texttt{http://www.boost.org/}. 

\bibitem{dot} \textbf{The DOT Language}. \textit{http://www.graphviz.org/doc/info/lang.html}.
 
\end{thebibliography}



