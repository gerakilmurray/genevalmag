\chapter{Conclusión y trabajos futuros}
\label{chap:conclusiones}

\minitoc

En el desarrollo del presente capítulo se presentarán comentarios finales del proyecto y además, posibles extensiones y trabajos a futuro.

\section{Conclusión}

En esta sección se expondrán las conclusiones obtenidas luego del desarrollo de este proyecto.\\

Durante el desarrollo de este trabajo se ha estudiado e introducido conocimientos sobre \textit{Gramáticas de Atributos} (GA), desde el punto de vista de definiciones, como así también de problemáticas y estado actual de de las mismas. Dentro de las GA, se ha trabajado con una familia, nueva y en desarrollo, como lo son las Multiplanes (MAG), presentadas por Wuu Yang (1998). Una de las características principales de estas GAs radica en su poder expresivo  y además, la posibilidad de desarrollar evaluadores estáticos mediante técnicas que se apoyan en secuencias de visitas (evaluadores orientado a visita con múltiples pasadas). 

El aporte y contribución principal de este trabajo, es el desarrollo de una herramienta, denominada \maggen, que automáticamente genera evaluadores estáticos para MAG.

En la siguiente sección se abordarán aspectos importantes relacionados a los resultados obtenidos sobre \maggen. Finalmente se puede decir que, los autores, aparte de tener en mente la culminación de la carrera, nunca perdieron la motivación y entusiasmo en el desarrollo de la herramienta de una manera eficiente.

\section{Resultados obtenidos}

Como conclusiones específicas sobre \maggen, se obtuvo una herramienta modularizada, eficiente y completamente desarrollada en C++, que cumplió los objetivos propuestos tanto de parte del grupo de desarrollo, como de la directiva del proyecto. Además de que no se conocen herramientas que trabajen sobre MAG, en este sentido, \maggen\ adquiere mayor utilidad. Se pueden resumir las características de \maggen\ en los siguientes puntos:

\begin{itemize}
\item El lenguaje de especificación de \maggen\ para una AG tiene sintaxis transparente, concisa y amplia, facilitando la traducción de cualquier gramática de atributos.

\item La herramienta fue desarrollada íntegramente en C++, como así también el evaluador generado. 

\item La construcción de secuencias de visitas se logro mediante un algoritmo simple y eficiente que genera la mínima cantidad de secuencias visita.

\item Tanto el algoritmo de contrucción de planes, como el de secuencia de visitas mantienen propiedades de compatibilidad y consistencia con los grafos de dependencia aumentados (ADP).

\item La construcción de planes se efectúa exhaustivamente sólo sobre las reglas alcanzables de la gramática.


 \item El evaluador generado es \textit{estático}, lo que garantiza que no hay procesamiento ``overhead'' en la evaluación, ya que el proceso de cómputo de planes de evaluación fue realizado por \maggen.
 
 \item La generación de código produce una representación compacta de cada estructura necesaria para la evaluación. Un ejemplo de ello, es que los planes no son generados en su totalidad, sino sólo sus contextos, esto debido a que la evaluación se realiza a través de las secuencias de visita.

\end{itemize}

\section{Extensiones}
Algunas de las posibles extensiones que se tienen en cuenta para \maggen\ se detallan a continuación:
\begin{itemize}
\item Extender \maggen\ para gramáticas dentro de la familia $NC(K)$.

\item Desarrollo de bibliotecas de funciones de utilidad de uso común en especificaciones, 
como manejo de ambientes, algoritmos de análisis estático (interpretación abstracta) y
generación de código.
\item Extensión de especificaciones, al estilo de las GA orientadas a objetos con mecanismos como redefinición o delegación (forwarding) de reglas para permitir un lenguaje de especificación más modular.
\item Introducción de sentencias de acceso no local de atributos para simplificar las
especificaciones.

\end{itemize}

\section{Trabajos futuros}
Entre los principales trabajos a futuro para \maggen\ se encuentran:
\begin{itemize}

\item Implementar la generación de código al estilo ``\textit{plugins}''. Lo que permitiría extensiones varias, transparentes y elegantes para generar código en diferentes lenguajes. Esto implicaría la definición de una API del motor de generación de código.

\item Definir una API para la construcción de los AST de entrada al evaluador generado, lo que permitiría que herramientas externas puedan generar los AST entrada (al estilo A-terms).
% Este puntos podría ser analizado teniendo en cuenta los A-Terms como ejemplo.

% \item Pruebas de rendimiento como comparaciones con otras herramientas similares.
% (Silver, definir sintaxis y semántica, es modular, puede que genere evaluador dinámico)

\item Permitir definir atributos de \textbf{alto orden}, es decir, atributos que pueden ser un árbol. Para los cuales, su evaluación supondría un nuevo proceso de igual complejidad que el necesario para decorar al AST de entrada.

\item Permitir definiciones de plantillas de atributos para lograr mayor genericidad en las especificaciones.

\end{itemize}


\maggen\ se encuentra disponible en \urllink{http://genevalmag.googlecode.com/}