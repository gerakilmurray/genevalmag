\chapter{Conclusión}
\label{chap:conclusiones}

\minitoc

En el desarrollo del presente capítulo se presentaran comentario finales del proyecto y además, posibles extensiones y trabajos a futuro.

\section{Conclusión}

En esta sección nos dedicaremos a exponer las conclusiones obtenidas luego del desarrollo de este proyecto.\\

Durante todo el desarrollo de este trabajo hemos estudiado e iniciado los conocimientos sobre \textit{gramáticas de atributos}, desde el punto de vista de definiciones, como así también de problemáticas y estado actual de desarrollo de las mimas. Además, hemos trabajado con una familia de GA, relativamente nuevas, como lo son las Multi-plans (MAG), presentas por Wuu Yang (1998). En consecuencia de este trabajo, se pudo desarrollar \maggen, como herramienta generadora de evaluadores para dicha gramáticas, considerada como el aporte y contribución principal del trabajo realizado.

Como conclusiones especificas sobre \maggen, se obtuvo una herramienta modularizable, eficiente y completamente desarrollada en C++, que cumplió los objetivos propuestos tanto de parte del grupo de desarrollo, como de la directiva del proyecto. Además de que no se conocen herramientas que trabajen sobre MAG, en este sentido, \maggen\ toma un auge aún mayor.\\

Finalmente podemos decir que, los autores, aparte de tener en mente la culminación de la carrera, nunca perdieron la motivación y entusiasmo en el desarrollo de la herramienta de una manera eficiente.

% \section{Aportes}
% 
% bla bla

\section{Trabajos futuros}
Como trabajos a futuro o extensiones de \maggen\ se encuentran:
\begin{itemize}
\item BLALA
\item BLALA
\end{itemize}