\chapter{Apéndice}
\label{chap:appendix}

\section{Especificación completa en \spirit}
\label{append:grammarspirit}

\lstinputlisting[basicstyle=\tiny, numbers=left, numberstyle=\tiny, numbersep=5pt, language=c++]{input_file_code/full_grammar.cpp}

\section{Ejemplo: Gramática de Atributos Multiplan presentada por Wuu Yang}
\label{append:agwuuyang}

\subsection{Pseudocódigo}
\begin{lstlisting}[basicstyle=\scriptsize, escapeinside=@@, backgroundcolor=\color{white}]
     (R1)   S @$\rightarrow$@ XYZ      
              S.s0 := X.s1 + Y.s2 + Ys3 + Zs4
              X.i1 := Y.s3  
              Y.i2 := X.s1
              Y.i3 := Y.s2
     (R2)   Y @$\rightarrow$@ m        
              Y.s2 := Y.i2
              Y.s3 := 1
     (R3)   Y @$\rightarrow$@ n        
              Y.s2 := 2
              Y.s3 := Y.i3
     (R4)   X @$\rightarrow$@ m        
              X.s1 := X.i1
     (R5)   Z @$\rightarrow$@ Y        
              Z.s4 := Y.s3
              Y.i2 := 3
              Y.i3 := Y.s2
\end{lstlisting} 

\subsection{Con lenguaje de especificación}

\subsubsection*{Archivo de input: \texttt{ag\_wuu\_yang.input}}
\lstinputlisting[basicstyle=\tiny, numbers=left, numberstyle=\tiny, numbersep=5pt, language=specmag]{input_file_code/ag_wuu_yang.input}

\section{Código generado por \maggen: Ejemplo Wuu Yang.}
\label{append:agwuuyangcode}

Los archivos fueron obtenidos invocando a \maggen\ de la siguiente forma:

\begin{center}
\footnotesize
\texttt{\maggen\ -fo ../Out\_wuu\_yang -o maggen -f  ../examples/ag\_wuu\_yang/ag\_wuu\_yang.input}
\end{center}

\subsection*{Archivo interface: \texttt{maggen.hpp}}

\lstinputlisting[basicstyle=\tiny, numbers=left, numberstyle=\tiny, numbersep=5pt, language=c++]{input_file_code/maggen.hpp}

\subsection*{Archivo implementación: \texttt{maggen.cpp}}

\lstinputlisting[basicstyle=\tiny, numbers=left, numberstyle=\tiny, numbersep=5pt, language=c++]{input_file_code/maggen.cpp}

\normalsize