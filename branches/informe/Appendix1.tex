\chapter{Apéndice: Algoritmos y Ejemplos}
\label{chap:appendix}

\section{Especificación completa en \spirit}
\label{append:grammarspirit}

\lstinputlisting[basicstyle=\tiny, numbers=left, language=c++]{input_file_code/full_grammar.cpp}

\section{Ejemplo: MAG presentada por Wuu Yang}
\label{append:agwuuyang}

\subsection{Pseudocódigo}
\begin{lstlisting}[basicstyle=\scriptsize, backgroundcolor=\color{white}]
     (R1)   S `$\rightarrow$` XYZ      
              S.s0 := X.s1 + Y.s2 + Y.s3 + Z.s4
              X.i1 := Y.s3  
              Y.i2 := X.s1
              Y.i3 := Y.s2
     (R2)   Y `$\rightarrow$` m        
              Y.s2 := Y.i2
              Y.s3 := 1
     (R3)   Y `$\rightarrow$` n        
              Y.s2 := 2
              Y.s3 := Y.i3
     (R4)   X `$\rightarrow$` m        
              X.s1 := X.i1
     (R5)   Z `$\rightarrow$` Y        
              Z.s4 := Y.s3
              Y.i2 := 3
              Y.i3 := Y.s2
\end{lstlisting} 

\subsection{Con lenguaje de especificación de \maggen}

\subsubsection{Archivo de input: \texttt{ag\_wuu\_yang.input}}
\lstinputlisting[basicstyle=\tiny, numbers=left, columns=fullflexible, language=specmag]{input_file_code/ag_wuu_yang.input}

\subsubsection{Archivo de output: \texttt{Grammar\_mag.log}}
\lstinputlisting[basicstyle=\tiny, numbers=left, columns=fullflexible, language=specmag]{input_file_code/Grammar_mag.log}

\section{Código generado por \maggen: Ejemplo Wuu Yang.}
\label{append:agwuuyangcode}

Los archivos fueron obtenidos invocando a \maggen\ de la siguiente forma:

\begin{center}
\footnotesize
\texttt{\maggen\ -fo ../Out\_wuu\_yang -o maggen -f  ../examples/ag\_wuu\_yang/ag\_wuu\_yang.input}
\end{center}

\subsection{Archivo interface: \texttt{maggen.hpp}}
\label{append:maggenhpp}
\lstinputlisting[basicstyle=\tiny, columns=fullflexible, numbers=left, language=c++]{input_file_code/maggen.hpp}

\subsection{Archivo implementación: \texttt{maggen.cpp}}
\label{append:maggencpp}
\lstinputlisting[basicstyle=\tiny, columns=fullflexible, numbers=left, language=c++]{input_file_code/maggen.cpp}

\subsection{Archivo cabecera: \texttt{Node.hpp}}
\label{append:nodehpp}
\lstinputlisting[basicstyle=\tiny, numbers=left, columns=fullflexible, language=c++]{input_file_code/Node.hpp}

\subsection{Archivo cabecera: \texttt{Plan.hpp}}
\label{append:planhpp}
\lstinputlisting[basicstyle=\tiny, numbers=left, columns=fullflexible, language=c++]{input_file_code/Plan.hpp}

\normalsize
\chapter{Apéndice: Documentación completa de \maggen.}
\label{chap:appendix-b}
La documentación completa de \maggen\ puede ser encontrada en el CD que acompaña este documento. La misma cuenta con más de 400 paginas (Documentación generada con \textit{doxygen}), por lo que no fue conveniente incluirla en este documento. 

Además, de contar con detalles de implementación de cada clase, cuenta con diagramas auto-generados, que muestran el acoplamiento entre cada modulo.

