\documentclass[a4paper,11pt,twoside]{ThesisStyle}

% \usepackage[spanish]{babel} .
% \usepackage{ucs}
% \usepackage[utf8x]{inputenc} 

\usepackage[spanish]{babel}% division de silabas en spanish
\usepackage[utf8]{inputenc}

% \usepackage[section]{placeins}

\usepackage[format=plain,labelfont=bf,up,textfont=it,up]{caption}

\makeatletter
\renewcommand{\@pnumwidth}{1.75em}
\renewcommand{\@tocrmarg}{2.75em}
\makeatother

\usepackage[titletoc]{appendix}
\renewcommand\appendixtocname{Ap\'endices}
\renewcommand\appendixpagename{Ap\'endices}
% AMS Math
\usepackage{amsmath, amsthm, amssymb}
% \usepackage{amsmath,amssymb}
% \usepackage[latin1]{inputenc}

\usepackage{longtable}
\usepackage{multirow}

\usepackage{listings}
\usepackage{color}
\usepackage{textcomp}
\usepackage{subfigure}
\definecolor{listinggray}{gray}{0.9}
\definecolor{lbcolor}{rgb}{0.9,0.9,0.9}
\definecolor{gris}{rgb}{0.8,0.8,0.8}
\definecolor{azul}{rgb}{0.0,0.3,0.6}

\lstdefinelanguage{specmag}{
  keywords={compute, end, all, semantic, domain, attributes, rules, sort, op, function, infix, prefix, postfix, syn, inh, left, right, non_assoc, and, and_eq, asm, auto, bitand, bitor, break, case, catch, class, compl, const, const_cast, continue, default, delete, do, double, dynamic_cast, else, enum, explicit, export, extern, false, for, friend, goto, if, inline, long, mutable, namespace, new, not, not_eq, operator, or, or_eq, private, protected, public, register, reinterpret_cast, return, short, signed, sizeof, static, static_cast, struct, switch, template, this, throw, true, try, typedef, typeid, typename, union, unsigned, using, virtual, void, volatile, wchar_t, while, xor, xor_eq, bool, char, float, int, string
%   ,repeat, until, procedure, then, from
  },
  sensitive=true,
  morecomment=[s]{/*}{*/},
  morecomment=[l]{\//},
  morestring=[d]{"}
}

\usepackage{slashbox}
% \renewcommand{\baselinestretch}{0.97}
\lstset{
    backgroundcolor=\color{lbcolor},
    tabsize=2,
%   rulecolor=,
    rulesepcolor=\color{azul},
    language=C++,
    morekeywords={repeat, until, procedure, then, from,
                  lexeme_d, as_lower_d, longest_d,
                  \!, \+, \*, \>\>, \-, \&, \|, \=, \%,
                  anychar_p, alnum_p, alpha_p, digit_p, lower_p, upper_p, space_p, ch_p, str_p, oct_p, hex_p, uint_p, int_p, real_p, eps_p, end_p,
                  symbols, add,
                  end},
    basicstyle=\footnotesize,
%     basicstyle=\renewcommand{\baselinestretch}{0.97}\small\tt,
    upquote=true,
    aboveskip={1\baselineskip},
    columns=[c]fixed,
    showstringspaces=false,
    extendedchars=true,
    breaklines=true,
    prebreak = \raisebox{0ex}[0ex][0ex]{\ensuremath{\hookleftarrow}},
    frame=single,
    showtabs=false,
    showspaces=false,
    showstringspaces=false,
    identifierstyle=\ttfamily,
    keywordstyle=\color[rgb]{0.1,0.1,0.6}\bfseries,
    commentstyle=\color[rgb]{0.133,0.545,0.133},
    stringstyle=\color[rgb]{0.627,0.126,0.941},
    keepspaces=true,
    escapeinside=``,
    numbersep=5pt,
    numberstyle=\tiny
}

\usepackage{bookman}
\usepackage[T1]{fontenc}
% \usepackage[left=3cm,right=2cm,top=4cm,bottom=3cm,includefoot,includehead,headheight=13.6pt]{geometry}
\usepackage[left=1.3in,right=1.1in,top=1.1in,bottom=1.1in]{geometry}
\renewcommand{\baselinestretch}{1.2}

% Table of contents for each chapter
\usepackage[nottoc]{tocbibind}
\usepackage{minitoc}
\setcounter{minitocdepth}{4}
\setcounter{parttocdepth}{1}
\mtcindent=5pt
\renewcommand\mtctitle{Contenido}
\renewcommand\mlftitle{Figuras}
\renewcommand\mlttitle{Tablas}

\usepackage{aecompl}

% Glossary / list of abbreviations

% \usepackage[intoc]{nomencl}
% \renewcommand{\nomname}{Lista de abreviaturas}

\newcommand{\maggen}{\textbf{magGen}}
\newcommand{\boost}{\textit{\textbf{Boost}}}
\newcommand{\spirit}{\textit{\textbf{Spirit}}}
\newcommand{\flecha}{$\hfill \Longrightarrow \hfill$}
\newcommand{\textbtt}[1]{\texttt{\textbf{#1}}}

% \makenomenclature

% \usepackage{ifpdf}
% 
% \ifpdf
%   \usepackage[pdftex]{graphicx}
%   \DeclareGraphicsExtensions{.eps,.jpg}
%   \usepackage[a4paper,pagebackref,hyperindex=true]{hyperref}
% \else
%   \usepackage{graphicx}
%   \DeclareGraphicsExtensions{.ps,.eps}
%   \usepackage[a4paper,dvipdfm,pagebackref,hyperindex=true]{hyperref}
% \fi

\usepackage{graphicx}
\usepackage[a4paper,pagebackref,hyperindex=true]{hyperref}
\graphicspath{{.}{images/}}

% Links in pdf
\usepackage{color}
\definecolor{linkcol}{rgb}{0,0,0.4} 
\definecolor{citecol}{rgb}{0.5,0,0} 

% Change this to change the informations included in the pdf file

% See hyperref documentation for information on those parameters

\hypersetup
{
bookmarksopen=true,
pdftitle="Informe de \maggen",
pdfauthor="Kilmurray - Picco", 
pdfsubject="Generador de Evaluadores estáticos para MAG.", %subject of the document
pdftoolbar=false, % toolbar hidden
pdfmenubar=true, %menubar shown
pdfhighlight=/O, %effect of clicking on a link
colorlinks=true, %couleurs sur les liens hypertextes
pdfpagemode=UseNone, %aucun mode de page
pdfpagelayout=SinglePage, %ouverture en simple page
pdffitwindow=true, %pages ouvertes entierement dans toute la fenetre
linkcolor=linkcol, %couleur des liens hypertextes internes
citecolor=citecol, %couleur des liens pour les citations
urlcolor=linkcol %couleur des liens pour les url
}

% definitions.
% -------------------

\setcounter{secnumdepth}{3}
\setcounter{tocdepth}{2}

% Some useful commands and shortcut for maths:  partial derivative and stuff

% \newcommand{\pd}[2]{\frac{\partial #1}{\partial #2}}
% \def\abs{\operatorname{abs}}
% \def\argmax{\operatornamewithlimits{arg\,max}}
% \def\argmin{\operatornamewithlimits{arg\,min}}
% \def\diag{\operatorname{Diag}}
% \newcommand{\eqRef}[1]{(\ref{#1})}

% \usepackage{rotating}                    % Sideways of figures & tables
%\usepackage{bibunits}
%\usepackage[sectionbib]{chapterbib}          % Cross-reference package (Natural BiB)
%\usepackage{natbib}                  % Put References at the end of each chapter
                                         % Do not put 'sectionbib' option here.
                                         % Sectionbib option in 'natbib' will do.
\usepackage{fancyhdr}                    % Fancy Header and Footer

% \usepackage{txfonts}                     % Public Times New Roman text & math font
  
%%% Fancy Header %%%%%%%%%%%%%%%%%%%%%%%%%%%%%%%%%%%%%%%%%%%%%%%%%%%%%%%%%%%%%%%%%%
% Fancy Header Style Options

\pagestyle{fancy}                       % Sets fancy header and footer
\fancyfoot{}                            % Delete current footer settings

%\renewcommand{\chaptermark}[1]{         % Lower Case Chapter marker style
%  \markboth{\chaptername\ \thechapter.\ #1}}{}} %

%\renewcommand{\sectionmark}[1]{         % Lower case Section marker style
%  \markright{\thesection.\ #1}}         %

\fancyhead[LE,RO]{\small\bfseries\thepage}    % Page number (boldface) in left on even
% pages and right on odd pages
\fancyhead[RE]{\scriptsize\bfseries\nouppercase{\leftmark}}      % Chapter in the right on even pages
\fancyhead[LO]{\scriptsize\bfseries\nouppercase{\rightmark}}     % Section in the left on odd pages

\let\headruleORIG\headrule
\renewcommand{\headrule}{\color{black} \headruleORIG}
\renewcommand{\headrulewidth}{1.0pt}
\usepackage{colortbl}
\arrayrulecolor{black}

\fancypagestyle{plain}{
  \fancyhead{}
  \fancyfoot{}
  \renewcommand{\headrulewidth}{0pt}
}

\usepackage[plain]{algorithm}
\usepackage[noend]{algorithmic}

%%% Clear Header %%%%%%%%%%%%%%%%%%%%%%%%%%%%%%%%%%%%%%%%%%%%%%%%%%%%%%%%%%%%%%%%%%
% Clear Header Style on the Last Empty Odd pages
\makeatletter

\def\cleardoublepage{\clearpage\if@twoside \ifodd\c@page\else%
  \hbox{}%
  \thispagestyle{empty}%              % Empty header styles
  \newpage%
  \if@twocolumn\hbox{}\newpage\fi\fi\fi}

\makeatother
 
%%%%%%%%%%%%%%%%%%%%%%%%%%%%%%%%%%%%%%%%%%%%%%%%%%%%%%%%%%%%%%%%%%%%%%%%%%%%%%% 
% Prints your review date and 'Draft Version' (From Josullvn, CS, CMU)
% \newcommand{\reviewtimetoday}[2]{\special{!userdict begin
%     /bop-hook{gsave 20 710 translate 45 rotate 0.8 setgray
%       /Times-Roman findfont 12 scalefont setfont 0 0   moveto (#1) show
%       0 -12 moveto (#2) show grestore}def end}}
% You can turn on or off this option.
% \reviewtimetoday{\today}{Draft Version}
%%%%%%%%%%%%%%%%%%%%%%%%%%%%%%%%%%%%%%%%%%%%%%%%%%%%%%%%%%%%%%%%%%%%%%%%%%%%%%% 

% \newenvironment{maxime}[1]
% {
% \vspace*{0cm}
% \hfill
% \begin{minipage}{0.5\textwidth}%
% %\rule[0.5ex]{\textwidth}{0.1mm}\\%
% \hrulefill $\:$ {\bf #1}\\
% %\vspace*{-0.25cm}
% \it 
% }%
% {%
% 
% \hrulefill
% \vspace*{0.5cm}%
% \end{minipage}
% }

% \let\minitocORIG\minitoc
% \renewcommand{\minitoc}{\minitocORIG \vspace{1.5em}}



% \newenvironment{bulletList}%
% { \begin{list}%
% 	{$\bullet$}%
% 	{\setlength{\labelwidth}{25pt}%
% 	 \setlength{\leftmargin}{30pt}%
% 	 \setlength{\itemsep}{\parsep}}}%
% { \end{list} }


%\newenvironment{definition}[1][Definicion]{\begin{trivlist}
%\item[\hskip \labelsep {\bfseries #1}]}{\end{trivlist}}

\newtheorem{definition}{Definici\'on}[section]
\newtheorem{theorem}{Teorema}[section]


\newcounter{ruleAGcounter}
\newenvironment{AG}
{
 \setcounter{ruleAGcounter}{0}
 \begin{center}
 \begin{figure}[!ht]
}
{
 \end{figure}
 \end{center}
}

\newcommand{\newproduction}[2]{$p_\arabic{ruleAGcounter}$: 
            \emph{#1} $\rightarrow$ \emph{#2}
            \addtocounter{ruleAGcounter}{1}\\}

\newenvironment{Attribution}
{\hspace*{1.8cm}\textbf{attribution}\\}
{\hspace*{1.8cm}\textbf{end}\\}

\newcommand{\attribution}[1]{\hspace*{2cm}\emph{#1}\\}

\newcommand{\clausderiva} [1] { 
  \makebox[0pt][l]{ \raisebox{-1.2ex}{\hspace{3pt}\tiny{#1}} }
  \makebox[0pt][l]{ \raisebox{0.9ex}{\hspace{3pt}\tiny{*}} }
  \makebox[1.1\width]{$ \Longrightarrow $}
}

% \renewcommand{\epsilon}{\varepsilon}

% centered page environment

\newenvironment{vcenterpage}
{\newpage\vspace*{\fill}\thispagestyle{empty}\renewcommand{\headrulewidth}{0pt}}
{\vspace*{\fill}}

\newcommand{\urllink}[1]{\htmladdnormallink{#1}{#1}}

\newenvironment{items}{
\begin{itemize}
  \setlength{\itemsep}{1pt}
  \setlength{\parskip}{0pt}
  \setlength{\parsep}{0pt}
}{\end{itemize}}


\begin{document}

\newcommand{\HRule}{\rule{\linewidth}{0.6mm}}
\begin{titlepage}

\begin{center}

\sffamily{
  \Large{
    \textbf{Tesis de la carrera\\
            Licenciatura en Ciencias de la Computación\\
    }
  }
}

\vspace*{0.5cm}
\HRule \\
  {\Huge\textbf{\maggen\\}}
\vspace*{0.5cm}
  {\large{\textbf{\rmfamily{Generador de Evaluador Estático para Gramática de Atributos Multiplan}}}}
\\
 
\HRule \\[0.5cm]



\vspace*{1cm}

\Large{\textbf{Autores:}\\ \textbf{Kilmurray}, Gerardo Luis\\ \textbf{Picco}, Gonzalo Martín\\}
\vspace*{0.6cm}
{------\\}
\vspace*{0.6cm}
\textbf{Director:}\\ Mg. Arroyo Marcelo Daniel

\vspace*{1cm}
 \includegraphics[width=40px,height=60.2px]{unrc.jpg}\\
\vspace*{0.7cm}
\large{Departamento de Computación\\
       Facultad de Ciencias Exactas, Físico-Químicas y Naturales\\
       Universidad Nacional de Río Cuarto\\
       Córdoba - Argentina}
\end{center}
\end{titlepage}
\sloppy

\titlepage


\dominitoc

\pagenumbering{roman}

\cleardoublepage

\section*{Agradecimientos}

 A toda la familia.

\tableofcontents

\mainmatter

\chapter{Introducci\'on}
\label{chap:intro}
\minitoc

En las ciencias de la computación los lenguajes juegan un rol muy importante en muchas disciplinas. Desde los comienzos
se buscaron mecanismos para describirlos y manejarlos, esta rama de la informática ha logrado muchos avances en el tema.
Dentro de los mecanismos desarrollados, las gramáticas han logrado ocupar unos de los lugares más destacados. La continua 
evolución y descubrimientos de nuevas gramáticas necesitó que se organizaran, y no fue hasta 1956 que Noam Chomsky propuso
una jerarquía sobre las gramáticas.\\
Las clasificó en 4 niveles:\\

\begin{tabular}{| p{1cm} | p{3.5cm} | p{4cm} | p{3cm} |}
\hline
\multicolumn{1}{|>{\columncolor[rgb]{0.8, 0.8, 0.8}}l|}{\textbf{Tipo}} &
\multicolumn{1}{|>{\columncolor[rgb]{0.8, 0.8, 0.8}}l|}{\textbf{Lenguaje}} &
\multicolumn{1}{|>{\columncolor[rgb]{0.8, 0.8, 0.8}}l|}{\textbf{Autómata}} &
\multicolumn{1}{|>{\columncolor[rgb]{0.8, 0.8, 0.8}}l|}{\textbf{Restricciones}} \\ \hline

\textbf{0} & Recursivamente enumerable (LRE) & Máquina de Turing (MT) & Sin restricciones \\ \hline
\textbf{1} & Dependiente del contexto (LSC) & Autómata linealmente acotado & $\alpha A \beta \rightarrow \alpha\gamma\beta$ \\ \hline
\textbf{2} & Independiente del contexto (LLC) & Autómata con pila & $A \rightarrow \gamma$ \\ \hline
\textbf{3} & Regular (RL) & Autómata finito & $A \rightarrow aB$ \\
  &              &                 & $A \rightarrow a$ \\ \hline
\end{tabular}





Desde que D. Knuth introdujo en 1966 las gramáticas de atributos (GA), estas se han utilizado ampliamente para el desarrollo de herramientas de procesamiento de lenguajes formales como compiladores, intérpretes, traductores como también para especificar la semántica de lenguajes de programación. Las gramáticas de atributos son un formalismo simple para la especificación de la semántica de lenguajes formales, como los lenguajes de programación o de especificación. Integran la modularidad que brindan las gramáticas libres de contexto y la expresividad de un lenguaje funcional.

\section{Gramática de Atributos}

En una gramática de atributos, se relaciona con cada símbolo de una gramática libre de contexto un conjunto de atributos. Cada regla o producción tiene asociados un conjunto de reglas semánticas que toman la forma de asignación a atributos de valores denotados por la aplicación de una función, la cual puede tomar como argumentos intancias de atributos pertenecientes a los símbolos que aparecen en la producción.
Las reglas semánticas inducen dependencias entre los atributos que ocurren en la producción. El orden de evaluación es implícito (si existe) y queda determinado por las dependencias entre las instancias de los atributos.
Una regla semántica se podrá evaluar cuando las instancias de los atributos que aparecen como sus argumentos estén evaluadas. Un evaluador de gramáticas de atributos debe tener en cuenta las dependencias entre las instancias de atributos para seguir un orden consistente de evaluación de los mismos.
Si una GA contiene dependencias circulares no podría ser evaluada ya que no podría encontrarse un orden de evaluación. Existen numerosas herramientas basadas en este formalismo o en alguna de sus ex-tensiones, entre las cuales podemos mencionar yacc, Yet Another Compiler-Compiler,desarrollado por AT\&T, AntLR, JavaCC, JavaCUP, ELI y muchas otras.


\section{Arbol sintactico atribuido}


\section{Métodos de Evaluación}

Los métodos estáticos deben tener en cuenta todos los posibles árboles sintácticos posibles a ser generados por la gramática y calcular todas las posibles dependencias entre las instancias de los atributos. Además, se deberán detectar las posibles depen-dencias circulares, para informar la viabilidad de su evaluación.
Esto se conoce como el problema de la circularidad, el cual se ha demostrado ser intrínsecamente exponencial [20]. El problema de la circularidad ha motivado que muchos investigadores hayan rea-lizado esfuerzos en la búsqueda e identificación de familias o subgrupos de gramáticas de atributos, para las cuales puedan detectarse circularidades con algoritmos de menor complejidad (polinomial o lineal).
Estas familias imponen restricciones sobre la gramática de atributos o sobre las dependencias entre sus atributos para garantizar que una GA no sea circular, con el costo de restringir su poder expresivo. 
Las clases de familias de gramáticas de atributos que se han utilizado para el desarrollo de herramientas eficientes y que se encuentran ampliamente analizadas en la bibliografía especializada, encontramos las s-atribuidas2 , l-atribuidas, las gramáticas de atributos ordenadas (OAG) y las absolutamente no circulares (ANCAG)[2]. En 1980, Uwe Kastens[23] caracterizó las gramáticas de atributos ordenadas y propuso un método para su evaluación, denominado secuencias de visita. Estas son secuencias de operaciones que conducen el recorrido del árbol sintáctico atribuido y realizan la evaluación de las instancias de los atributos. Kastens propone un método para generar las secuencias de visita en tiempo poli-nomial para la familia OAG.
Mas recientemente, en 1999, se han propuesto nuevas familias de GA para las que se pueden implementar evaluadores eficientes basado en métodos estáticos y con un mayor poder expresivo que las utilizadas tradicionalmente[44].



\subsection{Evaluci\'on din\'amica}

bla bla

\subsection{Evaluaci\'on est\'atica}

\subsection{Evaluacion de la familia ANCAG}
\section{Secuencia de visita}
\section{Generaci\'on de evaluadores para GA bien definidas}
\section{Evaluaci\'on durante el parsing}

bla bla

 %introduccion
\chapter{Clasificación de gramática de atributos}
\label{chap: clas_ag}
\minitoc

La clasificación de las gramática de atributos puede verse teniendo en cuenta dos aspectos: según la \textbf{estrategia de evaluación} y según las \textbf{dependencias entre los atributos}. 

Cada una de las familias de gramática de atributos tiene su propio poder expresivo, como así también, sus restricciones en el método de evaluación. La relación entre estos dos aspectos esta dado por el siguiente razonamiento: \textit{mientras mas restricciones en el método de evaluación, menos poder expresivo}.

En el capítulo siguiente trataremos la familia de las gramática de atributos multi-planes (MAG) que son las usadas en el marco de este trabajo. A continuación trataremos otras familias que contribuyen a la teoría de gramática de atributos.

A continuación se presenta un teorema que describe la relación entre las
familias de GA presentadas en esta seción, en base a su poder expresivo.
Las familias por debajo de la línea divisoria se corresponden con estrategias de
evaluación preestablecidas.

\begin{table}
\begin{center}
\begin{tabular}{ll}
\textbf{AG}         & Gramáticas de atributos \\
\textbf{WDAG}       & Gramáticas de atributos bien definidas\\
\textbf{ANCAG}      & GA absolutamente no circulares \\
\textbf{EOAG}       & GA Ordenadas Extendidas \\
\textbf{OAG}        & GA Ordenadas \\
\hline
\textbf{m-APAG} & GA evaluables en \emph{m} pasadas alternantes \\
\textbf{n-PAG}      & GA evaluables en \emph{n} pasadas \\
\textbf{L-AG}       & GA l-atribuidas \\
\textbf{S-AG}       & GA s-atribuidas \\
\end{tabular}
\end{center}
\caption{Jerarquía de GA}
\label{jer-GA}
\end{table}

Es fácil ver desde sus correspondientes definiciones que $IDP-{ANCAG}(p) \subseteq EDP(p)$, 
para cada producción $p \in P$ de una GA, ya que cada grafo $IDP-{ANCAG}(p)$ está definido 
en base a las dependencias directas de los atributos de los símbolos que ocurren en $p$ mas las 
dependencias (transitivas) de las producciones que tienen a esos símbolos como 
parte izquierda, no en cualquier parte como en el caso de los grafos $EDP(p)$.

Por lo tanto, los grafos de la familia $ANCAG$ nunca contendrán arcos no
contenidos en los grafos correspondendientes de la familia $EOAG$, lo que 
implica claramente la inclusión de las familias, ya que si una GA es $ANCAG$ 
también es $EOAG$. \\

La inclusión de las familias definidas en base a una estrategia de evaluación es
obvia. Sólo cabe aclarar que una $L-AG$ es una $n-PAG$ con $n=1$ con una pasada
descendente de izquierda a derecha.


\section{Clasificación basada en la estrategia de evaluación}

hablar de las WDAG y también de las OAG. Tener en cuenta (Marcelo p 44)

\begin{itemize}
\item AG      Gramáticas de atributos
\item WDAG    Gramáticas de atributos bien definidas   
\item ANCAG   GA absolutamente no circulares
\item EOAG    GA Ordenadas Extendidas
\item OAG     GA Ordenadas       
\item m-APAG  GA evaluables en m pasadas alternantes       
\item n-PAG   GA evaluables en n pasada       
\item L-AG    GA l-atribuida        
\item S-AG    GA s-atribuidas
\end{itemize}

\section{Clasificación basada en dependencias}

bla bla

\section{Clasificación de Knuth}

cla cla

\subsection{Árbol sintáctico atribuido}


\subsection{Gramáticas no circulares (NC)}

bla bla

\subsection{ANCAG}
 %clasificacion de AG
\chapter{Gramática de Atributos Multi-planes}
\label{chap:mag}
\minitoc

Tal como lo presenta Wuu-Yang en \cite{wuu-yang1} la familia de \textit{gramática de atributos multi-planes} es una clase que se encuentra dentro de las WDAG y dentro de ellas de las NC.

La familia de las MAG es estrictamente mayor que las ANCAG. La importancia de esta familia radica en que el procedimiento de computación de planes de evaluación estáticos toma \textbf{tiempo polinomial} en el numero de símbolos y producciones.

A continuación trataremos en detalle la familia de las MAG y en el capitulo siguiente abordaremos el mecanismo de evaluación de las mismas.

\section{Gramática de atributos}
En esta sección, se define la notación que se usara en le desarrollo del presente capitulo para la definición de la familia MAG.
Básicamente, la notación usada es la utilizada por Wuu yang en \cite{wuu-yang1}, la cual proviene de la notación de Kastens en \cite{kastens}.

\begin{definition}
\label{def:grammarattr}
Una gramática de atributos es una tupla GA = (G, A, V, Dom, F, R) donde:
\begin{itemize}
\item G = (VN , VT , S, P ) es una gramática libre de contexto reducida y no ambigua.
\item A = $\cup_{X\in(VN \cup T)} A(X)$, es el conjunto finito de atributos (A(X) es el conjunto de atributos asociados al símbolo X)

\item V es el conjunto finito de dominios de valores de los atributos.
\item Dom : $A\rightarrow V$ asocia a cada atributo un dominio o conjunto de valores d ∈ V .
\item F es un conjunto finito de funciones semánticas de la forma:
\begin{equation}
f \subseteq (\bigotimes\limits_{j=0}^{k}{ Dom(a_{j} ))\rightarrow Dom(a_{0})}
\end{equation}

\item R = $\bigcup _{p∈P} R^{p}$ es el conjunto finito de reglas de atribución o ecuaciones asociadas a cada producción p ∈ P , donde
\begin{equation}
R^{p} = \bigcup\limits_{j=0}^{m^{p}}{\{r_{j}^{p}\}}\ \ \ \ \ \ (\#(R^{p} ) = m^{p} ≥ 0)
\end{equation}
y cada regla $r_{j}^{p} \in R^{p}$ , con 0 ≤ j ≤ $m^{p}$ es de la forma

\begin{equation}
r_{j}^{p}: X_{0}.a_{0} = f(X_{1}.a_{1} ,\dots , X_{k}.a_{k})
\end{equation} 
donde cada $X_{i}$ es un símbolo que ocurre en la producción \textit{p} , $a_{i} \in A(X_{i})$, ($0 \leqslant i \leqslant k$) y $f \in F$.

\end{itemize}
\end{definition}

Detalles de notación:
\begin{itemize}
\item Con respecto a CFG, en todos el desarrollo del trabajo se denotara a los símbolos no terminales con letras en mayúscula y a los símbolos terminales con letras en minúscula.
\item Se utilizará la notación \textbf{X.a} para significar que el atributo \texttt{a} está asociado al símbolo \texttt{X} (a $\in$ A(X)) y para denotar el valor de una ocurrencia o instancia del atributo \texttt{a} del símbolo \texttt{X} en una regla de atribución.

\end{itemize}
En \cite{tesismarcelo} (cap. 2) podemos ampliar con mayor detalle y formalismo lo analizado arriba.
\section{Preliminares}
\label{sec:pre-grafos}
\subsection{Grafo \textit{DP}}
Los grafos \textit{DP} denotan las relaciones de dependencias directas entre las instancias de la gramática. 

Un grafo \textit{DP} esta definido (\cite{estruc-algorit}) con las siguientes consideraciones: 

\begin{itemize}
\item Los nodos denotan \textit{instancias} de una producción.
\item Las aristas denotan la dependencia entre las instancias. Una arista $X_{i}.a\rightarrow X{j}.b$, \footnote{con \textit{i} y \textit{j} índices de ocurrencias consistentes con alguna ecuacion de la gramática} denota que la evaluación de la instancia \textit{$X_{i}.a$} depende de la evaluación de \textit{$Y_{i}.b$} 
\end{itemize}

El conjunto de dependencias directas de una producción, de la gramática, se denota como \textit{DP(p)}(\textit{p} producción de la gramática) y se define como:
\begin{definition}
Dada una producción p de una gramática de atributos definida como en \ref{def:grammarattr}, entonces
\begin{equation}
DP(p) = \{(X_{i}.a, X_{j}.b) | X_{i}.a \rightarrow Y_{j}.b \in R^{p} \}
\end{equation}
\end{definition}

\subsection{Grafo \textit{Down}}
Los grafos \textit{Down} denotan las relaciones de los atributos de un símbolo. 

Un grafo \textit{Down} esta definido (\cite{estruc-algorit}) con las siguientes consideraciones: 

\begin{itemize}
\item Los nodos denotan \textit{atributos} de un símbolo.
\item Las aristas denotan la dependencia entre los atributos de un símbolo. Dado los atributos $a$ y $b\in A(X)$ del símbolo $X\in VN$, una arista $a\rightarrow b$, denota que la evaluación del atributo \texttt{a} depende de la evaluación de \texttt{b}. 
\end{itemize}
El conjunto de dependencias entre los atributos de un símbolo se denota como  
\texttt{Down(X)}($X\in VN$ de una gramática G) y se define:
\begin{definition}
Dada un símbolo X de una gramática de atributos definida como en \ref{def:grammarattr}, entonces
\begin{equation}
Down(X) = \{(a,b) | a \rightarrow b \} con\ a,b \in A(X)
\end{equation}
\end{definition}
\subsection{Grafo \textit{DCG}}

\textit{DCG} significa \textit{downward characteristic graphs}, los mismos contienen las dependencias entre instancias de la gramática para una producción \textit{p}, teniendo en cuenta un símbolo en la gramática.
\begin{definition}
Dado q una producción de la forma $X_{0}\rightarrow \alpha_{0} X_{1} \alpha_{1} X_{2} \dots X_{k} \alpha_{k}$, el \textit{downward characteristic graph} of $X_{0}$ en los subárboles derivados vía la producción \textit{q}, denotado como $DCG_{X_{0}}(q)$, es un grafo donde: 
\begin{itemize}
\item Los nodos son atributos del símbolo $X_{0}$.
\item Una arista, $X.a \rightarrow X.b$, denota una dependencia (transitiva) de X.b sobre X.a en algún subárbol derivado desde $X_{0}$ vía \textit{q}.
\end{itemize}
\end{definition}
Tomemos el siguiente teorema presentado por Wuu-Yang en \cite{wuu-yang1}:
\begin{theorem}
$\bigcup\limits_{\textit{todo p}}{DCG_{X} (p) = Down (X)}$
\end{theorem}
\underline{Nota:} $DCG_{X}(p)$ contiene las dependencias, entre las instancias de la gramática, para el símbolo \texttt{X}, acotando el análisis para la producción \textit{p} y los posibles contextos inferiores.

En la seccion \ref{XXX} se analiza el algoritmo para contruir los grafos DCG.

\subsection{Grafo \textit{ADP}}

Las siglas \textit{ADP} significan \textit{augmented dependency graph}. El grafo \textit{ADP} esta definido por instancias de la gramática, en los nodos, y cada arista se define como: $X_{i}.a\rightarrow X{j}.b$, denota que la evaluación de la instancia \textit{$X_{i}.a$} depende de la evaluación de \textit{$Y_{i}.b$}.

El conjunto de dependencias aumentadas se denota como $ADP (q | p_{1}, p_{2}, \dots, p_{k})$ y se define:
\begin{definition}
Sea q una producción de la forma $X_{0}\rightarrow \alpha_{0} X_{1} \alpha_{1} X_{2} \dots X_{k} \alpha_{k}$. Sea $p_{i}$ una producción cuya parte izquierda es $X_{i}$ ($1\leqslant i \leqslant k$). 
\begin{equation}
ADP (q | p_{1}, p_{2}, \dots, p_{k}) = DP(q) \bigcup\limits_{k}^{i=1}{DGC_{X_{i}}} (p_{i})
\end{equation}
\end{definition}

A partir de la definición anterior surge la siguiente:
\begin{definition}
El conjunto de todas las posibles dependencias aumentadas para una producción q se define como:
\begin{equation}
SADP(q) = \bigcup\limits_{q\in P}{ADP (q | p_{1}, p_{2}, \dots, p_{k})} 
\end{equation}
\end{definition}

\section{Definición MAG}

Una gramática \textit{G} de la forma \ref{def:grammarattr} es una \textit{gramática de atributos multi-planes} si y solo si 
\begin{equation}
\forall q : q \in P: (\forall g:g\ es\ un\ grafo\ de\ q \wedge g \in SADP(q) : g\ es\ no\ circular) 
\end{equation}

Las \textit{gramáticas multi-plans} mantienen propiedades deseables como:
\begin{itemize}
\item La clases de gramáticas MAG cumplen la propiedad de que todas las gramáticas de atributos en dicha clase son \textbf{no circulares}.
\item Toda gramática MAG cumple con las propiedades de gramática bien definida, esto es, todo árbol sintáctico derivado sobre una MAG contiene dependencias no circulares.
\end{itemize}

De los conceptos analizado arriba surgen los siguientes teoremas:

\begin{theorem}
$IDP-ANCAG(q) = \bigcup SADP(q)$, para toda producción q. 
\end{theorem}

\begin{theorem}
Toda gramática ANCAG es una gramática MAG, pero no viceversa.
\end{theorem}

 %gramaticas de atributos multiplans
% \chapter{Introducci\'on}
\label{chap:met_eval}
\minitoc


bla bla

\section{Evaluci\'on din\'amica}

bla bla

\section{Evaluaci\'on est\'atica}

\subsection{Evaluacion de la familia ANCAG}
\subsection{Secuencia de visita}
\subsection{Generaci\'on de evaluadores para GA bien definidas}
\section{Evaluaci\'on durante el parsing}

bla bla

\begin{bulletList}
 \item First point
 \item Second point
% \item Here is an abbreviation reference \nomenclature{DTI}{Diffusion Tensor Imaging} DTI
\end{bulletList}

\chapter{Evaluación estática de MAG}
\label{chap:eval_est}
\minitoc

En este capítulo se presentan los algoritmos de construcción de cada tipo de grafo, así como los usados para el cómputo de planes de evaluación y sus respectivas secuencias de visita. En su mayoría, los mismos, se basan en los propuestos por Wuu Yang en \ref{wuu-yang1}. Mientras que el algoritmo para el cálculo de secuencias de visita fue íntegramente desarrollado e implementado en el marco de la tesis.

Para comprobar la correctitud de los algoritmos y lograr mayor grado de eficiencia, los mismos fueron testeados mediante la herramienta \textit{gcov} (Detalles en sección \ref{XXX}). La cual fue utilizada con mayor énfasis para el algoritmo de cómputo de secuencias de visita. En conjunto con los resultados propocionados por \textit{gcov} y análisis teóricos, la heurística que implementa el algoritmo fue refinada en varias oportunidades.

\section{Algoritmos para cómputo de grafos}

Todos estos algoritmos tienen una característica en común, expresan las depedencias existentes entre los atributos de una gramática. Por lo tanto en su construcción, los nodos y aristas son obtenidos de las relaciones entre dichos elementos.

\subsection*{Cómputo de DP}

Se debe generar un grafo DP por cada regla de la gramática. Es el más directo de construir, ya que por cada ecuación de la regla actual, se agrega una arista entre cada uno de los símbolos del \texttt{r-value} con destino el \texttt{l-value} de la ecuación. Obviamente se agregan los nodos necesarios y sin repeticiones. 

\begin{algorithm}[H]
\lstinputlisting[basicstyle=\scriptsize\ttfamily, language=specmag, numbers=left, columns=fullflexible]{input_file_code/compute_dp_wuu_yang}
\caption{\texttt{compute\_dp\_graphs}}
\end{algorithm}

\subsection*{Cómputo de Down}

Este tipo de grafo, se debe crear para cada uno de los símbolos no terminales de la gramática, considerando únicamente la relaciones de depedencia entre sus atributos.

\begin{algorithm}[H]
\lstinputlisting[basicstyle=\scriptsize\ttfamily, language=specmag, numbers=left, columns=fullflexible]{input_file_code/compute_down_wuu_yang}
\caption{\texttt{compute\_down\_graphs}}
\end{algorithm}

La función \texttt{project} lo que hace básicamente es dado un grafo, elimina todos los nodos y aristas relacionadas con esos nodos que no representen a los atributos que le pasan como parámetro.

\begin{algorithm}[H]
\lstinputlisting[basicstyle=\scriptsize\ttfamily, language=specmag, numbers=left, columns=fullflexible]{input_file_code/project_wuu_yang}
\caption{\texttt{project} sobre grafos}
\end{algorithm}

\subsection*{Cómputo de DCG}

Estos grafos se generan para cada regla de la gramática y considerando las depedencias entre los atributos del símbolo de la parte izquierda de la producción (\texttt{head}). Para computarlo se unen el grafo DP de la regla junto a todos los grafos Down de los símbolos de la parte derecha de la regla (\texttt{body}). Se utiliza nuevamente la función \texttt{project}.

\begin{algorithm}[H]
\lstinputlisting[basicstyle=\scriptsize\ttfamily, language=specmag, numbers=left, columns=fullflexible]{input_file_code/compute_dcg_wuu_yang}
\caption{\texttt{compute\_dcg\_graphs}}
\end{algorithm}

\subsection*{Cómputo de ADP}

Para generar estos grafos, Wuu Yang no plantea un algoritmo preciso aunque no es necesario ya que toda la heurística necesaria se encuentra en la definición de los mismos (ver \ref{mag:adpdef}).

Un grafo ADP corresponde a una regla bajo una cierta combinación de producciones para cada uno de sus hijos. Es decir, que para una misma regla pueden existir varios grafos ADP, ya que sus hijos pueden producir varios \textbf{contextos}\footnote{Ver \ref{XXX}.}.

\begin{algorithm}[H]
\lstinputlisting[basicstyle=\scriptsize\ttfamily, language=specmag, numbers=left, columns=fullflexible]{input_file_code/compute_adp_wuu_yang}
\caption{\texttt{compute\_adp\_graphs}}
\end{algorithm}

\section{Algoritmo de cómputo de planes}

Luego de verificar que la gramática es MAG\footnote{Ver \ref{XXX}.}, se debe generar los planes de evaluación para la gramática. Cada ADP producirá un plan de evaluación, por lo que ambos tendrán el mismo identificador, es decir, la regla a la cual pertenece y el contexto que se aplicó a los símbolos no terminales de la misma.

Este algoritmo propuesto por Wuu Yang, es detallado en profundidad mostrando su correctitud\footnote{Ver \ref{XXX} para más detalles.}.

La idea principal es utilizar una cola de trabajos, en la cual se insertan un pares que contienen una regla y un orden de evaluación para sus atributos. Para cada uno de los grafos ADP asociados a la regla, se le computa su orden de evaluación para el contexto especificado por el grafo.

Luego, para cada uno de los símbolos no terminales de la parte derecha de la regla, considerando la producción que le corresponde según el contexto, se obtiene el plan proyectado. Ese nuevo par, se intenta insertar en la cola de trabajos para seguir con el cómputo de planes.

\begin{algorithm}[H]
\lstinputlisting[basicstyle=\scriptsize\ttfamily, language=specmag, numbers=left, columns=fullflexible]{input_file_code/compute_plan_wuu_yang}
\caption{\texttt{compute\_plans}}
\end{algorithm}

La función \texttt{compute\_order}, debe agregarle al grafo ADP la información inducida por el orden para evaluar los atributos del símbolo \textbf{head} de la regla a la que pertenece el grafo.

Para ello, como todo orden se puede ver como una secuencia, se agregan al grafo un nodo por cada elementos del orden y las aristas para todos los pares posibles\footnote{Ver más detalles en \ref{XXX}}.

\begin{algorithm}[H]
\lstinputlisting[basicstyle=\scriptsize\ttfamily, language=specmag, numbers=left, columns=fullflexible]{input_file_code/compute_order_wuu_yang}
\caption{\texttt{compute\_order}}
\end{algorithm}

La funcionalidad que cumple \texttt{project}, es dado un grafo purgarlo de todo elemento ajeno a el conjunto de ocurrencias de atributos especificados como parámetro. Pero para evitar que al eliminar nodos y aritas el grafo pierda información sobre las depedencias entre los atributos, se calcula previamente la clausura transitiva del grafo.

\begin{algorithm}[H]
\lstinputlisting[basicstyle=\scriptsize\ttfamily, language=specmag, numbers=left, columns=fullflexible]{input_file_code/project2_wuu_yang}
\caption{\texttt{project} orden de evaluación}
\end{algorithm}

\section{Algoritmo de cómputo de secuencias de visita}

Las secuencias de visitas, son las traducciones de los planes de evaluación en secuencias compuestas por tres comandos\footnote{\texttt{compute}, \texttt{leave} y \texttt{visit}, ver más detalles en \ref{XXX}.} que permitan navegar por es AST a evaluar.

Este algoritmo ...

\section{Algoritmo de evaluación de atributos}

\begin{algorithm}[H]
\lstinputlisting[basicstyle=\scriptsize\ttfamily, language=specmag, numbers=left, columns=fullflexible]{input_file_code/eval_attribute_wuu_yang}
\caption{Evaluación de atributos}
\end{algorithm}

\begin{algorithm}[H]
\lstinputlisting[basicstyle=\scriptsize\ttfamily, language=specmag, numbers=left, columns=fullflexible]{input_file_code/traverse_wuu_yang}
\caption{\texttt{traverse}}
\end{algorithm} %evaluadores estaticos
\chapter{Metodolog\'ia de trabajo}
\label{chap:metodologia}
\minitoc


\section{Practicas de software}
El análisis y especificación de requerimientos puede parecer una tarea relativamente sencilla, pero la realidad es que el proceso de escribir un software requiere de un marco de trabajo para estructurar, planificar y controlar  el desarrollo del sistema.
Al mismo tiempo, el uso de herramientas en cada etapa del ciclo de vida (análisis, diseño, implementación y prueba), permite recorrer un camino de creación incremental del sistema, donde cada estadio del proceso refina el modelo. 
La importancia del uso de herramientas, modelos y métodos para asistir el proceso radica en visualizar y garantizar cualidades del producto desarrollado en practicas de software comprobadas teóricamente. 

Algunas de las prácticas de software se tratan en la siguiente sección:
\begin{description}
\item[\textbf{Análisis-Diseño}] Esta etapa se baso en el estudio del marco teórico compuesto por papers y libros propuestos por el director de tesis. Edemas, se concretaron reuniones frecuentes para evacuar dudas y tomar decisiones respecto a objetivos y aspectos a considerar en el modelo. Esta fase, también, se utilizo para familiarizarse y solidificar el manejo de herramientas empleadas en los distintos estadios del proceso. 

\item[\textbf{Implementación}] En la etapa de implementación se invirtió una gran porción del tiempo total del proyecto. Esta fase, se dividió principalmente, en abordar los distintos estadio considerados en el análisis-diseño, pero, también, el refinamiento del diseño era una tarea que jugaba un papel importante. 


\item[\textbf{Prueba}] Esta etapa esta íntimamente relacionada con la etapa anterior (implementación), debido a que fueron realizadas en conjunto. Es decir, las pruebas eran abordadas luego de la implementación de cada fase distinguida en análisis-diseño. Para ello se planteaba casos de prueba específicos para cada fase, basándose en casos abordados en el marco teórico soporte del proyecto.


\item[\textbf{Documentación}] La documentación ,a nivel de código, fue abordada desde la etapa de implementación hasta el fin del proceso. Esta etapa prioriza en hecho de clarificar detalles de implementación hacia la comunicación entre los desarrolladores, como así también para desarrolladores o posibles colaboradores externos al proyectos.
Edemas, en la parte final se utilizo full-time a la elaboración del informe, presentación y demas, que hacen el desarrollo de una tesina de materia de grado.
\end{description}

\section{Lenguaje de programación C++}
C++ es un lenguaje de programación con tipado estático, multi-paradigma, compilado y de propósito general. Fue desarrollado por Bjarne Stroustrup en el año 1979 en los laboratorios Bell, como una mejora al lenguaje de programación C, y fue originalmente llamado ``C con clases''.

El lenguaje ha evolucionado, ha sido estandarizado y aún continúa evolucionando. Actualmente, C++ soporta varios conceptos que permiten escribir programas con diferentes estilos: imperativo (procedural), orientado a objetos (herencia, polimorfismo, programación genérica, metaprogramación, etc).

La elección de C++ como lenguaje a utilizar en el desarrollo de \maggen surgió despues de reuniones con el director de tesis, en las cuales de evaluaron lenguajes y se analizaron parámetros como lo son: 

eficiencia en cuanto a tiempo de ejecución, posibilidad de redefinir de los operadores en un contexto dado, mediante la \textit{sobre carga de operadores}, utilización de librerías maduras como soporte de componentes necesarios y extras al objetivo de la tesis, etc. 

Esta ultima, permitió la disponibilidad de bibliotecas genéricas, como la STL (Standard Template Library) y Boost Library que fueron utilizadas para disponer de funcionalidades y estructuras con solidas referencias.


\section{Herramientas}
Las lista de herramientas que se detallan a continuación fueron utilizadas con resultados muy positivos en cada una de las etapas del desarrollo de sistema. Es de destacar que las herramientas son ``free software''.  
\begin{description}


\item [Eclipse] es un entorno de desarrollo integrado de código abierto multiplataforma para desarrollar lo que el proyecto llama ``Aplicaciones de Cliente Enriquecido'', opuesto a las aplicaciones ``Cliente-liviano'' basadas en navegadores. Esta plataforma, típicamente ha sido usada para desarrollar entornos de desarrollo integrados (del inglés IDE). La versión utilizada fue ``Galileo'' (lanzada el 24 de junio del 2009).   \footnote{ {\tt http://www.eclipse.org/}}

\item [Subversion] es un software de sistema de control de versiones. Es software libre bajo una licencia de tipo Apache/BSD y se le conoce también como \textit{svn} por ser ese el nombre del comando que se utiliza. Esta herramienta fue de vital importancia para llevar a cabo la coordinación, comunicación y elaboración controlada entre los desarrolladores autores del trabajo.\footnote{ {\tt http://subversion.apache.org/}}

\item [\LaTeXe] es una herramienta para sistema de composición de textos, orientado especialmente a la creación de libros, documentos científicos y técnicos que contengan fórmulas matemáticas. Este documento en su totalidad se escribió utilizando \LaTeXe.\footnote{ {\tt http://www.latex-project.org/}}

\item [kile] es un editor de Tex/LaTeX. Funciona conjuntamente con KDE en varios sistemas operativos.\footnote{ {\tt http://kile.sourceforge.net/}}

\item [Graphviz] es una herramienta de visualización de grafos de código abierto. Genera una gran variedad de formatos de salida \footnote{ {\tt http://www.graphviz.org/}}

\item [Nemiver] es una herramienta de debugger que se integra perfectamente en el entorno de escritorio GNOME. En la actualidad cuenta con un motor que utiliza el conocido GNU gdb debugger para depurar programas C/C++. \footnote{ {\tt http://projects.gnome.org/nemiver/}}

\item [Dia] es una herramienta para la creación de cualquier tipo de diagrama.\footnote{ {\tt http://live.gnome.org/Dia}}

\item[Bouml] es una herramienta para la creación de diagramas UML.\footnote{ {\tt http://bouml.free.fr/ }}

\item [Análisis estático de código] entre los que se encuentran:
      \begin{description}
      \item [Cloc] \textbf{C}ounter \textbf{l}ines \textbf{o}f \textbf{c}ode. contador de lineas en blanco, lineas de comentario y lineas de código reales en muchos lenguajes de programación. Cloc esta escrito en ''Perl`` sin dependencias externas fuera del estándar de la distribución ''Perl v5.6``.\footnote{ {\tt http://cloc.sourceforge.net/}}
      \item [CCCC] \textbf{C} and \textbf{C}++ \textbf{C}ode \textbf{C}ounter. Fue desarrollado como un campo de pruebas para una serie de ideas relacionadas con las métricas de software en un proyecto de Maestría \footnote{ {\tt http://cccc.sourceforge.net/}}
      \item [Gcov] es un test de cubrimiento o cobertura de código. Es una forma de probar partes del programa no incluidas en los casos de prueba. Se utiliza en conjunto con GCC. \footnote{ {\tt http://gcc.gnu.org/onlinedocs/gcc/Gcov.html}}
      \end{description}

\end{description} %metodologia de trabajo
\chapter{Acerca de \maggen}
\label{chap:disen_}
\minitoc

\section{\textquestiondown Qué es \maggen?}
\maggen\ es una herramienta generadora de evaluadores estáticos para gramáticas atribuidas multiplans(MAG). 

Esto es, dada una MAG en el lenguaje de especificación, \maggen\ computa los planes de evaluación y genera las distintas secuencias de visita para todos los posibles contextos. Estas secuencias, son codificadas en un archivo C++ junto con los algoritmos necesarios para evaluar cualquier AST perteneciente la gramática \textit{input} de \maggen.

El evaluador generado por \maggen\ recibe un AST y devuelve este AST decorado. El potencial del evaluador generado por \maggen, esta dado, por tener estáticamente todos las posibles secuencias de visita para los posibles AST de entrada, entonces sólo debe seleccionar las secuencias para el AST dado y luego recorrerlo según como las mimas lo indiquen.

\section{\textquestiondown Como funciona \maggen ?}
Dado una MAG, \maggen\ calcula los grafos de dependencias de atributos de los símbolos de la gramática. A partir de ellos, es posible la aplicación de algoritmos de para la obtención de planes de evaluación y luego secuencias de visitas de dichos planes. \maggen\ se basa en la construcción de 4 tipos de grafos, que se construyen de manera incremental sobre la gramática. Estos grafos, son los presentados en el capitulo XXX, Dependency graph (\textbf{DP}), Down graph (\textbf{Down}), Downward characteristic graph (\textbf{DCG}) y augmented dependency graph (\textbf{ADP}). Sobre estos últimos se basa el calculo de los planes de evaluación.

El funcionamiento de \maggen\ esta dado por la integración de 4 etapas, consideradas principales, que marcaron el proceso de desarrollo de la herramienta:
\begin{itemize}
\item Lenguaje especificación de MAG.
\item Parser del lenguaje,representación interna y chequeos.
\item Construcción de grafos y aplicación de algoritmos de computo de planes y secuencias de visita.
\item Generación de código.
\end{itemize}

El computo de \maggen\ se realiza atravesando cada una de estas etapas secuencialmente, es decir, la terminación exitosa de una, habilita la siguiente; por lo tanto cada etapa mantiene su salida de errores de manera independiente. 

La salida normal de \maggen que indica que se han realizado todas las etapas correctamente, es la siguiente:

\begin{lstlisting}[backgroundcolor=\color{white}]
               * Parsing grammar ---------- [  OK  ]
               * Generate graphs ---------- [  OK  ]
               * Build plans -------------- [  OK  ]
               * Build visit sequence ----- [  OK  ]
               * Generation code ---------- [  OK  ]

               Generation complete in: 0.372814 seconds.
\end{lstlisting}

En caso de funcionamiento anormal de alguna de las etapas, se detectará un \textbf{FAIL} en la etapa correspondiente y se indicará la información del mismo.

Un caso alternativo de salida, se da cuando \maggen\ detecta planes cíclicos, visualizándose un \textbf{ABORT} en la etapa de creación de planes:

\begin{lstlisting}[backgroundcolor=\color{white}] 
               * Parsing grammar ---------- [  OK  ]
               * Generate graphs ---------- [  OK  ]
               * Build plans -------------- [ ABORT ]

               ERROR: One o more graph ADP has an cycle in its dependencies.Look the folder GenEvalAG/Out_Gen_Mag for more details.

               Generation complete in: 0.056179 seconds.
\end{lstlisting}
Notar que, el mensaje de error indicara la ruta donde fueron almacenados los grafos con problemas de ciclicidad. Para este caso particular fue: \textit{GenEvalAG/Out\_Gen\_Mag.}

En lo que resta del capitulo se abordaran detalles de cada etapa, analizada en esta sección.

\section{Lenguaje de especificación de las MAG}

El lenguaje de especificación utilizado para la descripción de una Gramática de atributos (MAG) fue definido en el marco de este proyecto. Esto permite definir una gramática de atributos como input de \maggen.
 
La secciones que conforman la descripción de una gramática de atributos se corresponden con las características que definen a una gramática de atributos como tal (ver capítulo de GA).
 
Informalmente, el lenguaje de especificación se conforma de las siguientes partes:

\begin{description}
\item [Bloque Dominio Semántico] Destinando a la declaración de sort, operadores y funciones que se utilizarán en los otros dos bloques. Este bloque es denominado ``\texttt{semantic domain}''.
\item [Bloque de Atributos] Destinando a la declaración y definición de los atributos asociados a cada símbolo. Este bloque es denominado ``\texttt{attributes}''.
\item [Bloque de Reglas] Destinado a la declaración y definición de las reglas sintácticas de la gramática con sus correspondientes ecuaciones semánticas para cada atributo asociado a cada símbolo. Este bloque es denominado ``\texttt{rules}''.
\end{description}

A los tres bloques analizados anteriormente, podemos clasificarlos en dos, teniendo en cuenta su comportamiento o funcionalidad dentro de la especificación. Los dos primeros, son bloques puramente declarativos o dedicados a la definición de elementos que serán utilizados en el tercer bloque. Este bloque, es considerado el de mayor auge, ya que marca la sintaxis y semántica de la gramática.

Cada bloque del lenguaje de especificación contiene su sintaxis propia para su definición, es por ello que, en las secciones siguientes nos encargaremos de mostrar detalles de cada uno de ellos.

El análisis de cada bloque se realizará de una manera más formal y observando cada bloque como partes de una gramática.

Entonces, sea \textbf{G: CFG} que define el lenguaje de especificación para el archivo de entrada aceptado por \maggen. Se define la siguiente regla de \textbf{G: CFG} para el símbolo inicial ``S''.

\begin{lstlisting}[frame=shadowbox, rulesepcolor=\color{blue},language=inform, linewidth=10cm ]
S   =   'semantic domain' decl_Sd
    |   'attributes' decl_attrs
    |   'rules' decl_rules
\end{lstlisting}

En las secciones siguientes se presentaran los símbolos \textit{decl\_Sd}, \textit{decl\_attr} y \textit{decl\_rules} con más detalle.  

\subsection{Bloque Dominio semántico}

El bloque ``\texttt{semantic domain}'' es el encargado de la definición de elementos que serán necesarios para los bloques siguientes. 

El bloque semántico esta subdividido en 3 secciones, que se corresponden con la definición de sort, operadores y funciones, cada una con su sintaxis propia. 

Entonces se define el símbolo \textit{decl\_Sd} como:

\begin{lstlisting}[frame=shadowbox, rulesepcolor=\color{blue},language=inform,linewidth=10cm]
decl_Sd = (decl_sort)*
        | (decl_operator)*
        | (decl_function)*
\end{lstlisting}

El uso de ``\texttt{*}'' (``0 o mas veces'') en cada subsección y no de ``\texttt{+}'', esta dado, debido a que cada una de estas secciones son opcionales, es decir, no se obliga a la existencia de cada sección. Por ejemplo, podría interesar la definición de una gramática que no cuente con funciones.

La sintaxis particular de cada sección se analiza individualmente a continuación.

\subsubsection{Declaración de sort}
La subsección de ``\texttt{sort}'' declara todos los posibles tipos que serán necesarios para las declaraciones siguientes. Todo \texttt{sort} es distinguible en el lenguaje mediante un nombre.

A continuación se define el símbolo \texttt{decl\_sort}

\begin{lstlisting}[escapeinside=<>, frame=shadowbox, rulesepcolor=\color{blue},language=inform, linewidth=10cm]
decl_sort = 'sort' NAME_SORT  list_sort ';'

list_sort = ',' NAME_SORT list_sort
          | <$\lambda$> 
\end{lstlisting}

\textit{``NAME\_SORT''} representa el nombre del sort o tipo. El mismo, se corresponde con la definición de un identificador en el común de los lenguajes de programación. Es decir, acepta caracteres alfanuméricos y guiones bajos, restringiendo a los caracteres numéricos como primer carácter, como también palabras reservadas definidas por la especificación (para mas detalles ver ANEXO  XXX de implementación en \spirit).\\
% \ref{append:grammarspirit}

\underline{Ejemplo:} \begin{center}
    \fbox{\texttt{\ sort int;\ }}\end{center}
\vspace{0.2cm}
Esta línea declara el tipo ``int''.

\subsubsection*{Tipos predefinidos por el lenguaje}
\label{sec:typepredefined}

El lenguaje de especificación contempla los siguiente tipos básicos:

\begin{description}
\item [int] Tipo entero de 32 bits.

\item [float] Tipo real en punto flotante de 32 bits.

\item [bool] Tiene en cuenta los valores \texttt{true} y \texttt{false}.

\item [char] Tipo char en el común de los lenguajes (encerrado entre comillas simples).

\item [string] Cadenas de caracteres (entre comillas dobles).
\end{description}

En caso de declaración explícita de cualquiera de ellos, en la especificación, la línea no es reflejada en el funcionamiento interno de \maggen.

\subsubsection{Declaración de operadores}
La sección destinada a la declaración de \texttt{operadores} acepta 3 tipos de operadores, los cuales, difieren en su forma de uso y cantidad de operandos. Denominados, infijo, prefijo y posfijo. 

La interpretación de cada uno, esta dada por la interpretación natural común a todos los lenguajes de programación, a modo de ejemplo se muestran un operador de cada tipo para evitar problemas en esta sección:

\begin{itemize}
\item \underline{Operador prefijo:} \textit{Ejemplo:} -2. Operador de menos unario. 

\item \underline{Operador posfijo:} \textit{Ejemplo:} i++. Operador de auto-incremento en lenguaje C y C++.

\item \underline{Operador infijo:} \textit{Ejemplo: 2 + 3}. Operador ``suma''.
\end{itemize}

A continuación se define el símbolo \texttt{decl\_operator}:

\begin{lstlisting}[escapeinside=<>, frame=shadowbox, rulesepcolor=\color{blue}, language=inform ]
decl_operator = 'op' 'infix'  mode_op NAME_OP ':'
                NAME_SORT ','  NAME_SORT '->' NAME_SORT ';' 
              | 'op' 'prefix' mode_op NAME_OP ':'
                NAME_SORT '->' NAME_SORT ';'               
              | 'op' 'postfix' mode_op NAME_OP ':'
                NAME_SORT '->' NAME_SORT ';'
              | 'op' mode_op NAME_OP ':'
                NAME_SORT '->' NAME_SORT ';'

mode_op = '(' m_op ')'
        | <$\lambda$>

m_op    = NUM_PRECEDENCE ',' assoc
        | '_' ',' assoc
        | NUM_PRECEDENCE ',' '_'
        | '_' ',' '_'

assoc   = (left | right | non-assoc) 
\end{lstlisting}

\textit{``NAME\_OP''} representa un identificador para el operador (infija, prefija y posfija). Las restricciones y detalles a tener en cuenta para este identificador son las mismas que se analizaron para ``NAME\_SORT''.

\textit{``NUN\_PRECEDENCE''} representa un número positivo que define la precedencia del operador. Cabe aclarar que a mayor número mayor la precedencia.

\subsubsection*{Consideraciones}

Es importante analizar el uso de ``\_'' para precedencia y asociatividad en el hecho de que estos datos son tomados opcionalmente, es decir se puede omitir dicha información. Lo mismo sucede con el tipo del operador (infijo, prefijo y posfijo). 

Para estos casos especiales se utilizarán los siguientes valores por defecto:

\begin{description}
\label{desc:default}
\item [Precedencia] = \texttt{USHRT\_MAX}.

\item [Asociatividad] = \texttt{left}.

\item [Tipo de operador] = \texttt{prefix}.
\end{description}

Otro caso a tener cuenta es el uso de \texttt{non-assoc} como asociatividad del operador. Este caso define que el operador no tiene asociatividad, con lo que el uso del mismo en las ecuaciones debe respetar esta condición, en caso contrario se observará un error por mal uso.
\subsection*{Ejemplos}

\begin{enumerate}

\item 
\begin{center}
\fbox{\texttt{\ op infix (\_,right) *: int, int -> int;\ }}\end{center}
\vspace{0.2cm}
Esta línea declara el operador infijo ``\texttt{*}'' con precedencia por defecto y asociatividad \texttt{right}. Esta línea también podría haber sido definida como:
\vspace{0.2cm}
\begin{center}
\fbox{\texttt{\ op infix *: int, int -> int;\ }}\end{center}
\vspace{0.2cm}
donde se usan valores por defecto para precedencia y asociatividad.

\item 
\begin{center}
\fbox{\texttt{\ op prefix (60,non-assoc) \%: int -> int;\ }}\end{center}
\vspace{0.2cm}
Esta línea declara el operador prefijo ``\texttt{\%}'' con precedencia \texttt{60} y asociatividad \texttt{non-assoc}. Esta línea también podría haber sido definida como:
\vspace{0.2cm}

\begin{center}
\fbox{\texttt{\ op (60, non-assoc) \%: int -> int;\ }}\end{center}
\vspace{0.2cm}
y en el caso que se desee usar valores de asociatividad y precedencia por defecto así:
\vspace{0.2cm}

\begin{center}
\fbox{\texttt{\ op \%: int -> int;\ }}\end{center}
\end{enumerate}
\subsubsection{Declaración de funciones}
La noción de funciones dentro de la especificación es tomada con la noción natural de función matemática. Es decir, toda función esta definida mediante un identificador, un dominio y una imagen.

Definimos \texttt{decl\_function} como:

\begin{lstlisting}[escapeinside=<>,frame=shadowbox, rulesepcolor=\color{blue}, language=inform ]
decl_function = 'function' NAME_FUNC':' domain '->' NAME_SORT ';'

domain        = NAME_SORT ',' domain
              | NAME_SORT 
              | <$\lambda$>
\end{lstlisting}

\textit{``NAME\_FUNC''} define el identificador de la función, en el cual se asumen las mismas restricciones tomadas para los identificadores analizados en las secciones anteriores. Cabe aclarar que se acepta un dominio vacío lo que permite el uso de funciones que solo retornan un valor.

Es importante tener en cuenta que las funciones son tomadas con los valores por defecto de operador para asociatividad y precedencia \ref{desc:default}.\\

\underline{Ejemplo:}\ \begin{center}
\fbox{\texttt{\ function f:int, int, int, int -> real;\ }}                                                                           \end{center}
\vspace{0.2cm}
Esta línea declara la función ``\texttt{f}'' que tiene como entrada 4 elementos de tipo ``\texttt{int}'' y como salida un elemento de tipo ``\texttt{real}''.

\subsection{Bloque de Atributos}
En esta sección presentaremos el bloque ``attributes'' en detalle. La información que define un atributo dentro del lenguaje esta dado por: 

\begin{description}
\item [Nombre:] representa el nombre del atributo, el mismo respeta los requisitos de identificador analizados anteriormente.

\item [Clase de atributo:] está dado por la clase del atributo, esto es sintetizado (\texttt{syn}) o heredado (\texttt{inh}).

\item [Tipo:] está dado por el tipo del atributo. El mismo corresponde a un tipo básico (ver \ref{sec:typepredefined}) o a un sort definido en la sección de \textit{Sort}.

\item [Símbolos de pertenencia:] hace referencia a los símbolos a los cuales se asocia el atributo.
\end{description}

A continuación se define el símbolo \texttt{decl\_attrs} como:

\begin{lstlisting}[escapeinside=@@, frame=shadowbox, rulesepcolor=\color{blue}, language=inform]
decl_attrs = (d_attr)+ 

d_attr = NAME_ATTR ':' '<' c_attr '>' NAME_SORT 'of' symbols;

symbols = '{'list_symbol'}' 
        | 'all'
        | 'all' '-' '{' list_symbol '}'

c_attr = 'inh'
       | 'syn'
       | @$\lambda$@

list_symbol = SYMB_NON_TERMINAL ',' list_symbol
            | SYMB_NON_TERMINAL 
\end{lstlisting}

\textit{``NAME\_ATTR''} define el identificador de un atributo. Se tienen las mismas consideraciones que para el identificador de sort, operador y función.

\textit{``SYMB\_NON\_TERMINAL''} describe un símbolo no terminal de la gramática. En este punto se debe tener en cuenta que los símbolos utilizados deben ser símbolos no terminales utilizados en el bloque de reglas.

\textit{``NAME\_SORT''} declara el tipo del atributo, el mismo esta dado por un sort definido por el usuario o por un tipo predefinido por el lenguaje \ref{sec:typepredefined}.

\subsubsection*{Consideraciones}

\begin{itemize}
\item En la declaración de los símbolos a los cuales pertenece el atributos, ``\texttt{all}'' se interpreta como ``todos los símbolos'', es decir, el atributo declarado se asocia a todos los símbolos de la gramática. Además es posible utilizar el operado ``diferencia'' de conjuntos ``\texttt{-}'' para especificar el conjunto de símbolos a los cuales perteneces el atributo, como una expresión.

\item Si no se especifica la clase del atributo (sintetizado o heredado) el mismo es tomado como el caso por defecto a sintetizado.
\end{itemize}

\subsection*{Ejemplos}
\begin{enumerate}

\item 
\begin{center}
\fbox{\texttt{\ lex: syn <string>\ of all - {T};\ }}\end{center}
\vspace{0.2cm}
Se define el atributo ``\texttt{lex}'' sintetizado de tipo ``\texttt{string}'' para todos los símbolos excepto para el símbolo ``\texttt{T}''.

\item

\begin{center}
\fbox{\texttt{\ type: inh <string>\ of all;\ }}\end{center}
\vspace{0.2cm}
Se define el atributo ``\texttt{type}'' heredado de tipo ``\texttt{string}'' para todos los símbolos de la gramática.

\item 

\begin{center}
\fbox{\texttt{\ grade: <int>\ of {E, T};\ }}\end{center}
\vspace{0.2cm}
Se define el atributo ``\texttt{grade}'' sintetizado (por defecto) de tipo ``\texttt{int}'' para los símbolos ``\texttt{E}'' y ``\texttt{T}''.\\
\end{enumerate}
\subsection{Bloque de reglas}
Por último el bloque de reglas. La interpretación de las reglas dentro del lenguaje esta dada por la definición de gramática libre de contexto.          

El símbolo terminal se considera un símbolo entre comillas simples (\texttt{'}).\\ 

\underline{Ejemplo:}\ \fbox{\ \texttt{'literal'}\ }\\
\vspace{0.2cm}

Las ecuaciones describen las reglas semánticas que definen la sintaxis de la gramática. Cada ecuación define la interpretación semántica de los atributos de cada símbolo; esta definición se realiza mediante una expresión constituida por \textbf{instancias}, \textbf{literales} o aplicación de operadores y funciones a subexpresiones.
En este punto se deben tener en cuenta los requisitos necesarios de una gramática bien definida (Ver capítulo XXX).

A continuación se define el símbolo \texttt{decl\_rules}:

\begin{lstlisting}[frame=shadowbox, rulesepcolor=\color{blue}, language=inform]
decl_rules = (d_rule)+ 

d_rule = SYMB_NON_TERMINAL '::=' rigft_symb decl_eqs

rigft_symb = (SYMB_NON_TERMINAL | SYMB_TERMINAL)+

decl_eqs = 'compute' d_eqs 'end;'
         | ';'

d_eqs = instance '=' right_eq ';'

right_eq = leaf
         | leaf OP_NAME leaf right_eq
         | (OP_NAME)+ leaf
         | leaf (OP_NAME)+
         | NAME_FUNC '(' right_eq ')' 

leaf = instance
     | LITERAL

instance = SYMB_NON_TERMINAL '[' NUM_INS ']' '.' NAME_ATTR
            
\end{lstlisting}

\textit{``SYMB\_NON\_TERMINAL''} y \textit{``SYMB\_TERMINAL''} describen símbolos no terminales y terminales respetivamente. En este punto, se tiene en cuenta la diferenciación entre estos tipos de símbolos como se analizó en el párrafo anterior.
 
\textit{``OP\_NAME''} y  \textit{``NAME\_FUNC''} describen identificadores de operadores infijos, prefijos y posfijos (según su uso) y el de función respectivamente. Tanto los operadores como las funciones se asumen definidos en la sección ``\texttt{semantic domain}''.

\textit{``NUM\_INS''} corresponde con un numero mayor que cero que identifica la ocurrencia del símbolo (para mas detalle ver \ref{subsec:consirule} punto 4).

\textit{``LITERAL''} describe los tipos de literales entero, real, carácter, string y bool con las siguientes consideraciones:

\begin{description}
\item [Entero] es considerado un número entero de 32 bits. 
\item [Real] es considerado un número en punto flotante de 32 bits. La separación de decimales se da mediante el punto (.).
\item [Carácter] es considerado un carácter cualquiera entre comillas simples (').
\item [String] es considerado una cadena de caracteres entre comillas dobles (`` '').
\item [Bool] representa los valores \texttt{true} y \texttt{false}.
\end{description}

\subsubsection*{Consideraciones}
\label{subsec:consirule}
\begin{enumerate}
\item Es posible definir una regla de la gramática sin sección de ecuaciones, para ello se debe omitir la sección de ``\texttt{compute}'' en la definición. Si se define la sección de ``\texttt{compute}'' el lenguaje obliga a definir \textbf{al menos una ecuación}.

\item La sintaxis para el uso de los operadores esta dada por el tipo del operador: infijo, prefijo o posfijo. Para las funciones se utiliza la manera natural de invocación de una función en el común de los lenguajes de programación. Esto es, mediante el nombre de la función y los parámetros entre paréntesis.

\item La asociatividad y precedencia de la expresión en las ecuaciones es calculada mediante los valores definidos en las secciones correspondientes. Cabe aclarar que es posible utilizar paréntesis para agrupar subexpresiones.

\item La asociación del símbolo no terminal con el atributo se denomina \textbf{instancia}. La misma se corresponde con la ocurrencia del símbolo. 

Se compone de tres partes:
\begin{itemize}
\item Símbolo no terminal.
\item Índice sintáctico, comenzando de 0 para la primera ocurrencia.
\item Atributo.
\end{itemize}
\begin{center} \underline{Ejemplo:}\  
\fbox{\ E[2].type\ }\end{center} 

Asocia al símbolo ``E'' con el atributo ``type'' en la ocurrencia 3 del símbolo.

\end{enumerate}

\subsubsection*{Ejemplos}
\begin{enumerate}

\item 
\begin{center}

\fbox{\ E ::= E '+' E \ compute\ 
                       E[0].valor = E[1].valor + E[2].valor; 
                       end;\ }\end{center}

El ejemplo muestra una regla del símbolo ``E'' que contiene una ecuación que define el atributo \texttt{valor}. Sin consideramos, una gramática bien definida (ver cap XXX), \texttt{valor} es un atributo sintetizado de ``E''. Otra consideración a tener en cuenta es la enumeración sintáctica de cada símbolo; en este ejemplo cada símbolo ``E'' presenta su índice sintáctico que asocia a los atributos con cada símbolo. Es decir, cada símbolo ``E'' es reconocido como un nuevo símbolo. A modo de aclaración, podríamos definir la ecuación de la siguiente, obteniendo el mismo poder expresivo para la regla:

\begin{center}
\fbox{\ E ::= T '+' F \ compute\ 
      E[0].valor = T[0].valor + F[0].valor; 
      end;\ }\end{center}

Cabe aclarar que \texttt{valor} debe ser atributo de ``T'' y ``F''.
\vspace{0.2cm}

\item

\begin{center}
\fbox{\ digit ::= '1';\ }\end{center}

Se observa una regla para el símbolo no terminal ``digit'' sin ecuaciones semánticas. 
\vspace{0.2cm}
\end{enumerate}
\subsection{Comentarios}
La especificación permite agregar comentarios. Para una mejor familiarización con el usuario se han utilizado las mismas reglas sintácticas que C y C++ para el adicionado de líneas o bloques de comentarios. Las cuales se detallan a continuación:

\begin{description}
\item [$\textbf{/*}$ comment $\textbf{*/}$] es la forma de inserción de bloques de comentarios.
\item [$\textbf{//}$ line commet] es comentario de una línea.
\end{description} 

\subsection{Ejemplo}

En el apéndice \ref{append:agwuuyang} observamos uno de los casos de prueba desarrollado para la fase de testing de \maggen. El mismo es presentado en pseudocódigo y luego escrito en el lenguaje de especificación. 

Una característica que motivó el desarrollo de este ejemplo es que se trata de una gramática MAG pero no ANCAG. Además, el mismo, es un caso de estudio analizado en una de las principales bases teóricas que han sido usadas para el desarrollo de \maggen[referencia bibliográfica paper]. 

En la presentación del ejemplo en el lenguaje de especificación, se observa en principio (línea 1 a 7) un bloque de comentario y luego los bloques que definen la gramática como se ha analizado en las secciones anteriores; bloque semántico de la línea 8 a 13 luego el de atributos y a partir de la línea 25 el bloque de reglas con sus respectivas ecuaciones.

 
\section{Parser del lenguaje,representación interna y chequeos}
El parser del lenguaje de especificación se llevó a cabo utilizando las librerías \boost\  \spirit. Detalles de este tema, son tratado puntualmente en CAPITULO IMPL. 

En esta sección nos encargaremos de algunas consideraciones del parser desde el punto de vista de su funcionamiento, algunos detalles en la representación interna y de los chequeos realizados sobre la entrada de \maggen\ para asegurar características de gramática bien definida.
\subsection*{Parser del lenguaje}
El parser de la entrada de \maggen\ esta dado por un recorrido secuencial y ante cualquier error sintáctico, la herramienta finaliza su ejecución indicando el error correspondiente. 

Supongamos la siguiente porción de una entrada de \maggen:

\begin{lstlisting}[numbers=left, numberstyle=\tiny, numbersep=5pt, firstnumber=8, language=cobol, linewidth=12cm]
  ...
/****************************
 * Block of Semantic Domain *
 ****************************/
semantic domain
    sort ecuacion, lista;
    /*********************
     * List of Operators *
     *********************/
    op infix    10, left) +: int, int -> int;
  ...
\end{lstlisting}
La linea 17 presenta un error de sintaxis (falta de paréntesis), entonces \maggen\ finaliza el parser e informa de la siguiente manera:
\begin{lstlisting}[backgroundcolor=\color{white}, linewidth=15cm]
     * Parsing grammar ---------- [ FAIL ]

     Generation complete in: 0.015683 seconds.
     ERROR: Parsing Failed, the following text will not be able to parse:

     File: ./examples/ag_count/ag_count.input
     Line: 17
     Col:  5
\end{lstlisting}
Notar que, la linea y columna donde se indica el error corresponde al inicio de la zona donde la interpretación fue fallida. Podría suceder en varios casos que el error no se detecte específicamente en la linea y columna que se indica, entonces lo conveniente es buscar el error a partir del punto especificado. Para este caso particular, se indica linea:17 y columna:5, este punto se encuentra sobre \texttt{op}, pero el error se observa seguido de \texttt{infix}.

\subsection*{Representación interna}
Las estructuras creadas para la representación interna de la gramática se encuentran dentro de los paquetes \texttt{Attr\_grammar} y \texttt{Expression\_tree} de \maggen. Detalles de implementación sobre estos paquete serán tratando en la sección XXX (imp). 

Ahora nos encargaremos de algunas consideraciones respecto del diseño. 
\texttt{Expression\_tree} agrupa módulos para la representación de las expresiones de las ecuaciones y \texttt{Att\_grammar}, encapsula los demas componentes de la gramática.
\begin{figure}\centering
\includegraphics[width=450pt,height=180pt]{Disen.png}
\caption{\label{fig:disen}Interacción de Attr\_grammar en \maggen}
\end{figure}

El funcionamiento de \maggen\ esta dado por una instancia de la estructura llamada \texttt{Attr\_grammar}, la cual encapsula toda la información de la gramática de atributos parseada. La misma contiene los elementos que se agrupan en los paquetes anteriormente nombrados y es la que desencadena el calculo de planes, secuencias de visita y, por ultimo, la generación de código.  
En la figura \ref{fig:disen} se presenta un diagrama que muestra la interacción de esta estructura en el funcionamiento de \maggen.


\begin{figure}\centering
\includegraphics[width=450pt,height=297pt]{Disen2.png}
\caption{\label{fig:disen2}Diseño interno de una gramática de atributos en \maggen}
\end{figure}
Analicemos ahora el diseño de los paquetes nombrados arriba, y con ellos la estructura \texttt{Attr\_grammar}.  En la figura \ref{fig:disen2} se muestra un diagrama con la representación interna de una gramática dentro de \maggen. Notemos algunas observaciones del diseño que muestra el diagrama:
\begin{itemize}
\item La relación entre los paquetes se produce por un lado, mediante la clase \texttt{Rule}, en el sentido de que toda regla contiene un conjunto de ecuaciones y cada una de estas esta definida a través de un \textit{l\_value:}\texttt{Expr\_instance} y un \textit{r\_value:}\texttt{Expression}. Por otro lado, \texttt{Expr\_instance} se vincula con \texttt{Symbol} y \texttt{Attribute} y \texttt{Expr\_function} con \texttt{Function}.

\item Toda la información de una gramática de atributos esta representada en la clase \texttt{Attr\_grammar} y por ende, el ciclo de vida de cada de uno de sus componentes es administrado por esta clase, durante todo el funcionamiento de \maggen. 
\end{itemize}



\subsection*{Chequeos}

El conjuntos de chequeos que se realizan sobre la gramática de entrada a \maggen, están directamente relacionados con el concepto de gramática bien definida analizado en el cap XX. Además, se aplican una serie de chequeos extras para evitar cálculos inconsistentes en las etapas posteriores. La totalidad de los chequeos, son realizados sobre el bloque de reglas de la gramática\footnote{En su mayoría los chequeos se realizan sobre las ecuaciones de las reglas.}, debido a que este, es el más significativo y sobre el cual se apoyan los cálculos posteriores.

Los chequeos realizados son clasificados en sintácticos y semánticos, los primeros, en su mayoría, son realizados durante el parser, en cambio, los chequeos semánticos se realizan luego del parser de cada regla, aplicando escaneos específicos. De estos últimos nos encargaremos en esta sección.
El modulo que agrupa estos chequeos, se ubica en el paquete \texttt{Parser} de \maggen\ y se denomina \texttt{Semantics\_checks}.

A continuación se detallan dichos chequeos:

\begin{description}
\item [Precedencia] El chequeo de precedencia consiste en preservar la precedencia definida de cada operador. Dicho chequeo, es realizado sobre cada una de las ecuaciones definidas en la gramática. Por ejemplo:\\ Dada la siguiente ecuacion: 
\begin{center}
 \fbox{ E[0].valor = E[1].valor + E[2].valor * E[2].valor }
\end{center}
Su interpretación está dada por la definición de la precedencia de los operadores \texttt{+} y \texttt{*}(se asume que los dos operadores son infijos y están bien usado). Si tomamos \texttt{*} con mayor precedencia que \texttt{+}, \maggen\ interpreta implícitamente como si la ecuacion tuviera paréntesis:
\begin{center}
 \fbox{ E[0].valor = E[1].valor + (E[2].valor * E[2].valor) }
\end{center}

El uso de paréntesis en las expresiones define explícitamente la prioridad de los operadores, por lo que, \maggen\ asume esa precedencia. Por ejemplo en la ecuación vista arriba, si la definimos utilizando paréntesis, es decir:\  
 \fbox{E[0].valor = (E[1].valor + E[2].valor) * E[2].valor}\ la interpretación es tomada en ese sentido.

\item [Asociatividad] El chequeo de asociatividad realiza un análisis sobre las expresiones de las ecuaciones, para asegurar la asociatividad definida de cada operador. Por ejemplo, si se ha definido un operador \texttt{non-assoc}, se espera que el uso del mismo sea consistente con esa definición, es decir, no se considere una posible asociación del mismo. 

Cuando se utilizan paréntesis en la expresión, \maggen\ asume la asociatividad que se especifica explícitamente.

\item [Alcanzabilidad] Es una característica importante a tener en cuenta sobre los símbolos de la gramática. Este chequeo, realiza un análisis sobre la totalidad de las reglas en busca de símbolos no terminales no alcanzables desde el símbolo inicial. Este caso no es considerado error fatal en \maggen, sino como \textit{warning}.
\item [Condiciones de AG] Este chequeo tiene en cuenta un conjunto de condiciones que debe cumplir la gramática. Ellos son:
\begin{itemize}
\item Cada regla debe definir todos los atributos sintetizados del símbolo de la parte izquierda.
\item Cada regla debe definir todos los atributos heredados de los símbolos de la parte derecha.
\item Cada regla sólo define los atributos de los símbolos que participan en dicha regla.
\item En caso de redefinición alguna instancia en la regla, la misma es ignorada.
\item Los índices de las instancias deben ser consistente con la cantidad de apariciones del símbolo, en la regla. Ejemplo: \\En la siguiente regla:\ \fbox{ S ::= S T }\ los índices posibles de las instancias son \texttt{0} y \texttt{1} para el símbolo ``S'' y \texttt{0} para el símbolo ``T''.
\item La gramática debe respetar las condiciones exigidas para una gramática extendida. 
\end{itemize}
\end{description}

\section{Construcción de grafos y aplicación de algoritmos de computo de planes y secuencias de visita}
En la figura \ref{fig:disen} pudimos analizar el funcionamiento de \maggen, teniendo en cuenta la estructura \texttt{Attr\_grammar}, en esta sección abordaremos detalles en el diseño del paquete \texttt{Builder}, específicamente la construcción de grafos, de planes y de secuencias de visita. Un detalle a tener en cuenta es que, a partir de esta etapa,  todas las manipulaciones son realizadas sobre la sección de ecuacion de cada reglas.

\subsection*{Construcción de grafos}
Detalles de implementación serán analizando en la sección XX, ahora tomaremos un par de minutos para mencionar algunas consideración con respecto al funcionamiento de los grafos en \maggen.


La construcción de grafos se realiza en el modulo \texttt{builder\_graphs}. \maggen\ crea 4 tipos de grafos:
\begin{figure}\centering
 \includegraphics[width=200pt,height=77pt]{graph/dp.png}
\caption{\label{dpgraph} DP Graph.}
\end{figure}

\begin{figure}\centering
 \includegraphics[width=75pt,height=97pt]{graph/down.png}
\caption{\label{downgraph} DOWN Graph.}
\end{figure}

\begin{figure}\centering
 \includegraphics[width=200pt,height=101pt]{graph/dcg.png}
\caption{\label{dcggraph} DCG Graph.}
\end{figure}

\begin{figure}\centering
 \includegraphics[width=200pt,height=209pt]{graph/adp.png}
\caption{\label{adpgraph} ADP Graph.}
\end{figure}


\begin{itemize}
\item DP graph. Ver figura\footnote {\label{foot:graph} Figura generada utilizando la herramienta \textit{graphviz} sobre el archivo \texttt{.dot} generado por \maggen.} \ref{dpgraph}, generado a partir del ejemplo analizado en el apéndice \ref{append:agwuuyang}.
\item DOWN graph. Ver figura$^{\ref{foot:graph}}$ \ref{downgraph} generado a partir del ejemplo analizado en el apéndice \ref{append:agwuuyang}.
\item DCG graph.Ver figura$^{\ref{foot:graph}}$ \ref{dcggraph} generado a partir del ejemplo analizado en el apéndice \ref{append:agwuuyang}.
\item ADP graph. Ver figura$^{\ref{foot:graph}}$  \ref{adpgraph} generado a partir del ejemplo analizado en el apéndice \ref{append:agwuuyang}.
\end{itemize}

La construcción de cada tipo de grafo respeta los principios analizados en el capitulo XXX. 

Dentro de los 4 tipos de grafos analizados, los necesarios para el computo de etapas siguientes, son los ``ADP graph''. Los demas tipos de grafo, son requeridos como calculo previo para la construcción de los ADP.


\subsection*{Construcción de planes}
La etapa de construcción de planes, tal como lo muestra el diagrama de la figura \ref{fig:disen}, se realiza en el paquete ``Builder''. El calculo de planes en \maggen\ se basa en lo analizado en el cap XXX y teniendo en cuenta estos dos puntos:

\begin{enumerate}
\item El punto de entrada para el calculo de los planes esta dado sobre los grafos ADP. 
\item El calculo de planes se basa en un orden topológico de la evaluación de los atributos de los símbolos, teniendo en cuenta los distintos contextos posibles.
\end{enumerate}
En la sección XXX se tratan detalles de implementación, en donde podemos analizar el funcionamientos de lo que tratamos anteriormente, en especial el punto 2.


\subsection*{Construcción de secuencias de visita}
La etapa de construcción de secuencias de visita, tal como lo muestra el diagrama de la figura \ref{fig:disen}, se realiza en el paquete ``Builder''. Las mismas, son computados a partir de los diferentes planes obtenidos en el sección anterior.

El funcionamiento para la construcción de las secuencias, esta dado por la aplicación de una especie de \textit{simulación}, para la evaluación de cada atributo de los símbolos. El resultado de esta simulación, son los valores que conforman la secuencia de visita, donde la interpretación de ellos son: 
\begin{description}
\item [Compute] Valor mayor que cero que representa el numero de la ecuacion a resolver\footnote{La ecuación a computar pertenece a la regla del plan corriente.}.
\item [Visit] valor menor que cero que representa el numero nodo hijo a visitar\footnote{El nodo hijo esta dado por el contexto de la regla del plan corriente.}.
\item [Leave] valor ``0''.
\end{description}
A modo de ejemplo observemos las siguientes secuencias de visita generada por \maggen\ para el ejemplo analizado en el apéndice \ref{append:agwuuyang}:
\begin{description}
\item [\{-7,0,-8\}] Secuencia de visita para la regla \texttt{P3} (sin contexto inferior). En este caso, la secuencia de visita se traduce a lo siguiente:
\begin{enumerate}
\item \textbf{Computar} la ecuacion 7.
\item \textbf{Leave}\footnote{El leave retorna el control a la secuencia de visita de contexto superior, es decir, desde donde se invocó a esta secuencia.}.
\item \textbf{Computar} la ecuacion 8.
\end{enumerate}

\item [\{-11,1,-12,-10\}] Secuencia de visita para la regla \texttt{P5} con el contexto de \texttt{P2}. En este caso, la secuencia de visita se traduce a lo siguiente:
\begin{enumerate}
\item \textbf{Computar} la ecuacion 11.
\item \textbf{Visitar} el nodo hijo 1, es decir \texttt{Y}, como el contexto es \texttt{P2}, significa visitar a P2.
\item \textbf{Computar} la ecuacion 12.
\item \textbf{Computar} la ecuacion 10.
\end{enumerate}

\end{description}
Cabe aclarar que todas las secuencias de visita tienen una \textbf{leave} implícito al final, es decir, cuando se completo la computación de la secuencia se retorna a su ámbito de invocación.

\underline{Heurística de la simulación:}\\


Esta consiste en evaluar la ecuacion obteniendo sus dependencias de sus posibles contextos, entonces, de esta forma se obtienen las visitas a realizar para la computación de cada una de las instancias\footnote{Recordemos, en este punto, que una instancia es considerada una aparición del símbolo y que se compone de símbolo, índice y atributo.} de cada regla de la gramática.

\section{Generación de Código}
La etapa final de \maggen\ esta dado por la generación de código para el evaluador estático.  %diseño de magGen
\chapter{Detalles de Implementación de \maggen}
\label{chap:implem}
\minitoc

En esta sección se aclararan y expondrán decisiones que se realizaron a lo largo de la implementación de esta herramienta.

Primero, como ya se menciona en el capítulo anterior, el lenguaje sobre el cual se realizó el desarrollo de \maggen es C++, con lo que todo el código es principalmente característico de un modelo de programación Orientado a Objetos, aunque posee secciones de Programación Imperativa, para lograr ciertas optimizaciones y poder integrar componentes de externos a la herramienta.

\section{Parsing usando \boost\ \spirit}

El primer desafío de codificación, fue que se debía conseguir parsear el archivo de entrada de la herramienta, en el cual vendría la especificación de una Gramática de Atributos, respetando la sintaxis presentada anteriormente, o no.

La solución obtenida se apoya en la utilización de un framework reconocido mundialmente, denominado \spirit, perteneciente a la biblioteca de C++ llamada \boost. Esta decisión trajo dos grandes beneficios; la confiabilidad de el parser obtenido y la rápida obtención del mismo, ya que la gran ventaja de \spirit, es que permite escribir la definición de la gramática en lenguaje \textbf{C++}.

\spirit\ es un framework generador de analizadores sintácticos, o parsers, descendentes recursivos orientado a objetos implementado usando técnicas de meta-programación con plantillas. Las expresiones mediante plantillas, permiten aproximar la sintaxis de una ``\textit{\textit{Forma Backus Normal Extendida}}'' (\textbf{EBNF}) completamente en C++.

Se puede resumir que el proceso de análisis sintáctico, en este framework, se componente de cuatro partes.

\begin{figure}\centering
\includegraphics[width=400pt, height=172pt]{./spirit.png}
\caption{Procesos dentro de \spirit}\label{procesoSpirit}
\end{figure}

El usuario es el responsable de definir las ``\texttt{semantic actions}'' para lograr capturar los resultados intermedios, producidos durante la etapa de parsing y no sólo obtener un valor lógico sobre si se pudo o no, consumir toda la cadena de entrada.

Para comenzar a definir una gramática hay que crear una estructura que herede de la clase \texttt{grammar}. Dentro de ella se debe definir a su vez otra estructura templatizada denominada \texttt{definition}. En donde realmente estará la definición de la gramática.

\begin{center}\begin{lstlisting}[language=C++, basicstyle=\scriptsize,numbers=left, numbersep=5pt, numberstyle=\tiny]
struct my_grammar: public grammar <my_grammar>
{
    template <typename ScannerT>
    struct definition
    {
        rule <ScannerT> r;

        definition(my_grammar const& self)
        {
            r = /*... Aqui va la definicion ...*/;
        }

        rule <ScannerT> const& start() const
        {
            return r;
        }
    };
};
\end{lstlisting}\end{center}

Para definirla, disponemos de un amplio conjunto de herramientas. Primero, los operadores de metalenguaje listados en la tabla \ref{ope_spirit}, los cuales permiten construir nuevas reglas combinando reglas ya definidas, parsers y constantes permitidas en el framework.

\begin{figure}\centering
\begin{tabular}{| c | p{9cm} |}
\hline
\multicolumn{1}{|>{\columncolor[rgb]{0.8, 0.8, 0.8}}l|}{\textbf{Operador}} &
\multicolumn{1}{|>{\columncolor[rgb]{0.8, 0.8, 0.8}}l|}{\textbf{Semántica}} \\ \hline
A $=$                B & Definición de A en base a B \\ \hline
A $|$                B & Unión, acepta A o B, también llamada ``alternativa''\\ \hline
A $\&$               B & Intersección, acepta A y B \\ \hline
A $-$                B & Diferencia, acepta A pero no a B  \\ \hline
A $\textasciicircum$ B & Disyunción exclusiva, acepta A o B, pero no a ambos \\ \hline
A $>>$               B & Secuencia, acepta A seguido de B \\ \hline
A $\%$               B & Lista, acepta A separados por ocurrencias de B.\\
                       & Es equivalente a: A $>>$ *(B $>>$ A)\\ \hline
$*$                  A & Estrella de Kleene, 0 o más veces \\ \hline
$+$                  A & Positivo, 1 o más veces \\ \hline
$!$                  A & Opcional, 0 o 1 vez \\ \hline
\end{tabular}
\caption{Operadores de \spirit\ utilizados}\label{ope_spirit}
\end{figure}

Por otra parte, \spirit\ dispone de un gran conjunto de parsers que abarcan la mayoría de los tipos básicos, representación de datos y valores utilizados en el común de los lenguajes (ver tabla \ref{parsers}).

\begin{figure}\centering
\begin{tabular}{| l | l |}
\hline
\multicolumn{1}{|>{\columncolor[rgb]{0.8, 0.8, 0.8}}l|}{\textbf{Parser}} &
\multicolumn{1}{|>{\columncolor[rgb]{0.8, 0.8, 0.8}}l|}{\textbf{Entrada aceptada}} \\ \hline
anychar\_p & Cualquier carácter simple (incluyendo el carácter nulo: '$\setminus0$')\\ \hline
alnum\_p   & Caracteres alfa-numéricos \\ \hline
alpha\_p   & Caracteres alfabéticos \\ \hline
% blank\_p & Un espacio o tabulación \\ \hline
digit\_p   & Dígitos numéricos \\ \hline
lower\_p   & Caracteres en minúscula \\ \hline
upper\_p   & Caracteres en mayúscula \\ \hline
space\_p   & Espacios, tabulaciones, saltos de línea y nuevas líneas \\ \hline
ch\_p      & Carácter especificado como parámetro \\ \hline
str\_p     & Cadena especificada como parámetro \\ \hline
oct\_p     & Dígito en octal \\ \hline
hex\_p     & Dígito en hexadecimal \\ \hline
uint\_p    & Número entero sin signo de 32 bits\\ \hline
int\_p     & Número entero 32 bits\\ \hline
real\_p    & Número flotante 32 bits\\ \hline
eps\_p     & Cadena vacía (épsilon)\\ \hline
end\_p     & Carácter de fin de archivo (EOF)\\ \hline
\end{tabular}
\caption{\label{parsers}Parsers predefinidos de \spirit\ utilizados} 
\end{figure}

Además posee varias directivas que modifican el comportamiento de los parsers, encapsulándose en una expresión definida por el usuario (ver tabla \ref{directivas}).

\begin{figure}\centering
\begin{tabular}{| l | p{8cm} |}
\hline
\multicolumn{1}{|>{\columncolor[rgb]{0.8, 0.8, 0.8}}l|}{\textbf{Directiva}} &
\multicolumn{1}{|>{\columncolor[rgb]{0.8, 0.8, 0.8}}l|}{\textbf{Efecto}} \\ \hline
lexeme\_d    & Deshabilita la omisión de espacios en blanco (space\_p)\\ \hline
as\_lower\_d & Convierte en minúscula lo aceptado por la expresión\\ \hline
longest\_d   & Deshabilita el corto circuito, manda al analizador que pruebe todas las alternativas posibles y elija la secuencia más larga aceptada \\ \hline
\end{tabular}
\caption{\label{directivas}Directivas de \spirit\ aplicadas}
\end{figure}

Para el manejo de los símbolos válidos dentro de la definición de la gramática, \spirit soporta nativamente, el concepto de ``Tablas de símbolos''. Lo que permite registrar de manera dinámica nuevos símbolos, en particular dentro de nuestra herramienta nos ayudó a manejar los conjuntos de nombres válidos para los \texttt{sorts}, los símbolos no terminales permitidos en las ecuaciones, entre otros usos.

\begin{center}\begin{lstlisting}[language=C++, basicstyle=\scriptsize,numbers=left, numbersep=5pt, numberstyle=\tiny]
/*
 * Symbols's Table for the elements of an Attribute Grammar.
 */
symbols <> st_sorts;
symbols <> st_op_prefix;
symbols <> st_op_infix;
symbols <> st_op_postfix;
symbols <> st_functions;
symbols <> st_attributes;
symbols <> st_non_terminal;
\end{lstlisting}\end{center}

Para comenzar a interactuar con el framework, el usuario debe declarar un objeto de la estructura definida y pasarlo como parámetro a una función específica de \spirit.

\begin{center}\begin{lstlisting}[language=C++, basicstyle=\scriptsize, numbers=left, numbersep=5pt, numberstyle=\tiny]
my_grammar g;

if (parse(first, last, g, space_p).full)
{
    cout << "Parsing Succeeded\n";
}
else
{
    cout << "Parsing Failed\n";
}
\end{lstlisting}\end{center}

\subsection{\texttt{``semantic domain''} en \spirit}

En esta sección de la especificación se debían aceptar tres tipos de elementos: \texttt{sorts}, \texttt{operators} y \texttt{functions}.

Los \texttt{sorts} serán los tipos que se podrán utilizar para definir los dominios e imágenes de los operadores y funciones.

Según lo explicado en el Diseño, las reglas de cada clase, han sido codificados de la siguiente manera.

\begin{description}
\item [\texttt{sorts}] Exijimos que luego del identificador ``\texttt{sort}'' halla un espacio\footnote{Ver definición de space\_p \ref{parsers}}. El nombre leído, se usará para crear un nuevo Sort dentro de la GA y también se insertará en la tabla de símbolos de sorts. Se acepta una lista de nombres de sorts para comodidad del usuario.

\begin{center}\begin{lstlisting}[language=C++, basicstyle=\scriptsize, numbers=left, numbersep=5pt, numberstyle=\tiny]
/* Declaration of Sorts. */

r_decl_sort = lexeme_d[str_p("sort")>>space_p]>>
              (r_ident[&create_sort][st_sorts.add]%',')>>
              ';';
\end{lstlisting}\end{center}

\item [\texttt{operators}] Para los operadores también se exije el espacio. Por definición se aceptan tres tipos: infijos, prefijos e infijos, por lo que se hizo una regla para cada uno, sólamente discriminado los dominios, ya que todos poseen una sóla imágen. Notar que los sorts permitidos, son sacados directamente de las tablas de símbolos destinadas para ese propósito.

\begin{center}\begin{lstlisting}[language=C++, basicstyle=\scriptsize, numbers=left, numbersep=5pt, numberstyle=\tiny]
/* Declaration of Operators. */

r_decl_oper  = lexeme_d[str_p("op")>>space_p][&inic_func]>>
               (r_oper_infix|r_oper_postfix|r_oper_prefix)>>
               str_p("->")>>
               r_sort_st[&save_image_func]>>
               ';';

r_oper_infix = str_p("infix")[&save_mode_op]>>
               !r_oper_mode>>
               r_oper[&save_name_func][st_op_infix.add]>>
               ':'>>
               r_sort_st[&save_domain_func]>>','>>r_sort_st[&save_domain_func];

r_oper_postfix = str_p("postfix")[&save_mode_op]>>
                 !r_oper_mode>>
                 r_oper[&save_name_func][st_op_postfix.add]>>
                 ':'>>
                 r_sort_st[&save_domain_func];

r_oper_prefix = !(str_p("prefix")[&save_mode_op])>>
                !r_oper_mode>>
                r_oper[&save_name_func][st_op_prefix.add]>>
                ':'>>
                r_sort_st[&save_domain_func];

r_oper_mode = '('>>
               (uint_p[&save_prec_op]|'_')>>
               ','>>
               (r_oper_assoc[&save_assoc_op]|'_')>>
               ')';

r_oper_assoc = str_p("left")|"right"|"non-assoc";
\end{lstlisting}\end{center}

\item [\texttt{functions}] Para las funciones se permiten como dominio listas de sorts, los cuales serán agregados incrementalmente a la función que se está declarando.

\begin{center}\begin{lstlisting}[language=C++, basicstyle=\scriptsize, numbers=left, numbersep=5pt, numberstyle=\tiny]
/* Declaration of Functions. */

r_decl_func = lexeme_d[str_p("function")>>space_p]>>
              r_oper[&inic_func][&save_name_func][st_functions.add]>>
              ':'>>
              !r_dom_func>>
              str_p("->")>>
              r_sort_st[&save_image_func]>>
              ';';

r_dom_func = r_sort_st[&save_domain_func]%',';
\end{lstlisting}\end{center}

\end{description}


\subsection{\texttt{``attributes''} en \spirit }
sdsad


\subsection{\texttt{``rules''} en \spirit }

sadsadsd

\section{Algoritmo de generación de secuencia de visita}

bla bla
\section{Algoritmo de generación de código}
bla bla
 %Implementacion
\chapter{Usos}
\label{chap:usos}
\minitoc



\section{Uso de MagGen}
bla bla

\section{Uso del evaluador generado}
bla bla %usos de magGen
\chapter{Conclusión}
\label{chap:conclusiones}

\minitoc

En el desarrollo del presente capítulo se presentaran comentario finales del proyecto y además, posibles extensiones y trabajos a futuro.

\section{Conclusión}

En esta sección nos dedicaremos a exponer las conclusiones obtenidas luego del desarrollo de este proyecto.\\

Durante todo el desarrollo de este trabajo hemos estudiado e iniciado los conocimientos sobre \textit{gramáticas de atributos}, desde el punto de vista de definiciones, como así también de problemáticas y estado actual de desarrollo de las mimas. Además, hemos trabajado con una familia de GA, relativamente nuevas, como lo son las Multi-plans (MAG), presentas por Wuu Yang (1998). En consecuencia de este trabajo, se pudo desarrollar \maggen, como herramienta generadora de evaluadores para dicha gramáticas, considerada como el aporte y contribución principal del trabajo realizado.

Como conclusiones especificas sobre \maggen, se obtuvo una herramienta modularizable, eficiente y completamente desarrollada en C++, que cumplió los objetivos propuestos tanto de parte del grupo de desarrollo, como de la directiva del proyecto. Además de que no se conocen herramientas que trabajen sobre MAG, en este sentido, \maggen\ toma un auge aún mayor.\\

Finalmente podemos decir que, los autores, aparte de tener en mente la culminación de la carrera, nunca perdieron la motivación y entusiasmo en el desarrollo de la herramienta de una manera eficiente.

% \section{Aportes}
% 
% bla bla

\section{Trabajos futuros}
Como trabajos a futuro o extensiones de \maggen\ se encuentran:
\begin{itemize}
\item BLALA
\item BLALA
\end{itemize} %conclusion.
\appendix

\chapter{Apéndice}
\label{chap:appendix}

\section{Especificación completa en \spirit}
\label{append:grammarspirit}

\lstinputlisting[basicstyle=\tiny, numbers=left, language=c++]{input_file_code/full_grammar.cpp}

\section{Ejemplo: Gramática de Atributos Multiplan presentada por Wuu Yang}
\label{append:agwuuyang}

\subsection{Pseudocódigo}
\begin{lstlisting}[basicstyle=\scriptsize, escapeinside=@@, backgroundcolor=\color{white}]
     (R1)   S @$\rightarrow$@ XYZ      
              S.s0 := X.s1 + Y.s2 + Ys3 + Zs4
              X.i1 := Y.s3  
              Y.i2 := X.s1
              Y.i3 := Y.s2
     (R2)   Y @$\rightarrow$@ m        
              Y.s2 := Y.i2
              Y.s3 := 1
     (R3)   Y @$\rightarrow$@ n        
              Y.s2 := 2
              Y.s3 := Y.i3
     (R4)   X @$\rightarrow$@ m        
              X.s1 := X.i1
     (R5)   Z @$\rightarrow$@ Y        
              Z.s4 := Y.s3
              Y.i2 := 3
              Y.i3 := Y.s2
\end{lstlisting} 

\subsection{Con lenguaje de especificación}

\subsubsection*{Archivo de input: \texttt{ag\_wuu\_yang.input}}
\lstinputlisting[basicstyle=\tiny, numbers=left, language=specmag]{input_file_code/ag_wuu_yang.input}

\section{Código generado por \maggen: Ejemplo Wuu Yang.}
\label{append:agwuuyangcode}

Los archivos fueron obtenidos invocando a \maggen\ de la siguiente forma:

\begin{center}
\footnotesize
\texttt{\maggen\ -fo ../Out\_wuu\_yang -o maggen -f  ../examples/ag\_wuu\_yang/ag\_wuu\_yang.input}
\end{center}

\subsection*{Archivo interface: \texttt{maggen.hpp}}

\lstinputlisting[basicstyle=\tiny, numbers=left, language=c++]{input_file_code/maggen.hpp}

\subsection*{Archivo implementación: \texttt{maggen.cpp}}

\lstinputlisting[basicstyle=\tiny, numbers=left, language=c++]{input_file_code/maggen.cpp}

\normalsize


% \bibliographystyle{ThesisStyle}
% \bibliography{Thesis}

%\printnomenclature

\cleardoublepage
\begin{vcenterpage}
\noindent\rule[2pt]{\textwidth}{0.5pt}
\begin{center}
{\large\textbf{Design and Use of Numerical Anatomical Atlases for Radiotherapy\\}}
\end{center}
{\large\textbf{Resumen:}}
El tratamientos de lenguajes es uno de los temas mas estudiados en los ultimos años.
Las gramaticas de atributos son un formalismo que permite utilizar el poder descriptivo de las gramaticas libres de contexto y la expresividad de los lenguajes funcionales. 

{\large\textbf{Keywords:}}
Atlas-based Segmentation, non rigid registration, radiotherapy, atlas creation
\\
\noindent\rule[2pt]{\textwidth}{0.5pt}
\end{vcenterpage}

\end{document}
