\chapter{Clasificaci\'on de gramatica de atributos}
\label{chap: clas_ag}
\minitoc


La clasificacion de las gramatica de atributos puede verse teniendo en cuenta dos aspectos: segun la\textbf{ estrategia de evaluacion} y segun las \textbf{dependecias entre los atributos}. 

Cada una de las familias de gramatica de atributos tiene su propio poder expresivo, como asi tambien, sus restricciones en el metodo de evaluacion. La relacion entre estos dos aspectos esta dado por el siguiente razonamiento: \textit{mientras mas restricciones en el metodo de evaluacion, menos poder expresivo}.

En el capitulo siguiente trataremos la familia de las gramatica de atributos multi-plans (MAG) que son las usadas en el marco de este trabajo. A continuacion trataremos otras familias que contribuyen a la teoria de gramatica de atributos.

\section{Clasificaci\'on basada en la estrategia de evaluci\'on}

hablar de las WDAG y tambien de las OAG. TEner en cuenta (marcelo p 44)

        
AG      Gramáticas de atributos
WDAG    Gramáticas de atributos bien definidas   
ANCAG   GA absolutamente no circulares
EOAG    GA Ordenadas Extendidas
OAG     GA Ordenadas       
m-APAG  GA evaluables en m pasadas alternantes       
n-PAG   GA evaluables en n pasada       
L-AG    GA l-atribuida        
S-AG    GA s-atribuidas


\section{Clasificaci\'on basada en dependecias}

bla bla
\section{Clasificaci\'on de Knuth}

cla cla
\subsection{\'Arbol sint\'actico atribu\'ido}


\subsection{Gramáticas no circulates(NC)}
bla bla

\subsection{ANCAG}
