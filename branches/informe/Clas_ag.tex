\chapter{Clasificación de gramática de atributos}
\label{chap: clas_ag}
\minitoc

La clasificación de las gramática de atributos puede verse teniendo en cuenta dos aspectos: según la \textbf{estrategia de evaluación} y según las \textbf{dependencias entre los atributos}. 

Cada una de las familias de gramática de atributos tiene su propio poder expresivo, como así también, sus restricciones en el método de evaluación. La relación entre estos dos aspectos esta dado por el siguiente razonamiento: \textit{mientras mas restricciones en el método de evaluación, menos poder expresivo}.

En el capítulo siguiente trataremos la familia de las gramática de atributos multi-planes (MAG) que son las usadas en el marco de este trabajo. A continuación trataremos otras familias que contribuyen a la teoría de gramática de atributos.

A continuación se presenta un teorema que describe la relación entre las
familias de GA presentadas en esta seción, en base a su poder expresivo.
Las familias por debajo de la línea divisoria se corresponden con estrategias de
evaluación preestablecidas.

\begin{table}
\begin{center}
\begin{tabular}{ll}
\textbf{AG}         & Gramáticas de atributos \\
\textbf{WDAG}       & Gramáticas de atributos bien definidas\\
\textbf{ANCAG}      & GA absolutamente no circulares \\
\textbf{EOAG}       & GA Ordenadas Extendidas \\
\textbf{OAG}        & GA Ordenadas \\
\hline
\textbf{m-APAG} & GA evaluables en \emph{m} pasadas alternantes \\
\textbf{n-PAG}      & GA evaluables en \emph{n} pasadas \\
\textbf{L-AG}       & GA l-atribuidas \\
\textbf{S-AG}       & GA s-atribuidas \\
\end{tabular}
\end{center}
\caption{Jerarquía de GA}
\label{jer-GA}
\end{table}

Es fácil ver desde sus correspondientes definiciones que $IDP-{ANCAG}(p) \subseteq EDP(p)$, 
para cada producción $p \in P$ de una GA, ya que cada grafo $IDP-{ANCAG}(p)$ está definido 
en base a las dependencias directas de los atributos de los símbolos que ocurren en $p$ mas las 
dependencias (transitivas) de las producciones que tienen a esos símbolos como 
parte izquierda, no en cualquier parte como en el caso de los grafos $EDP(p)$.

Por lo tanto, los grafos de la familia $ANCAG$ nunca contendrán arcos no
contenidos en los grafos correspondendientes de la familia $EOAG$, lo que 
implica claramente la inclusión de las familias, ya que si una GA es $ANCAG$ 
también es $EOAG$. \\

La inclusión de las familias definidas en base a una estrategia de evaluación es
obvia. Sólo cabe aclarar que una $L-AG$ es una $n-PAG$ con $n=1$ con una pasada
descendente de izquierda a derecha.


\section{Clasificación basada en la estrategia de evaluación}

hablar de las WDAG y también de las OAG. Tener en cuenta (Marcelo p 44)

\begin{itemize}
\item AG      Gramáticas de atributos
\item WDAG    Gramáticas de atributos bien definidas   
\item ANCAG   GA absolutamente no circulares
\item EOAG    GA Ordenadas Extendidas
\item OAG     GA Ordenadas       
\item m-APAG  GA evaluables en m pasadas alternantes       
\item n-PAG   GA evaluables en n pasada       
\item L-AG    GA l-atribuida        
\item S-AG    GA s-atribuidas
\end{itemize}

\section{Clasificación basada en dependencias}

bla bla

\section{Clasificación de Knuth}

cla cla

\subsection{Árbol sintáctico atribuido}


\subsection{Gramáticas no circulares (NC)}

bla bla

\subsection{ANCAG}
