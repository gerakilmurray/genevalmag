\chapter{Gramática de Atributos Multi-planes}
\label{chap:mag}
\minitoc

Tal como lo presenta Wuu-Yang en \cite{wuu-yang1} la familia de \textit{gramática de atributos multi-planes} es una clase que se encuentra dentro de las WDAG y dentro de ellas de las NC.

La familia de las MAG es estrictamente mayor que las ANCAG. La importancia de esta familia radica en que el procedimiento de computación de planes de evaluación estáticos toma \textbf{tiempo polinomial} en el numero de símbolos y producciones.

A continuación trataremos en detalle la familia de las MAG y en el capitulo siguiente abordaremos el mecanismo de evaluación de las mismas.

\section{Gramática de atributos}
En esta sección, se define la notación que se usara en le desarrollo del presente capitulo para la definición de la familia MAG.
Básicamente, la notación usada es la utilizada por Wuu yang en \cite{wuu-yang1}, la cual proviene de la notación de Kastens en \cite{kastens}.

\begin{definition}
\label{def:grammarattr}
Una gramática de atributos es una tupla GA = (G, A, V, Dom, F, R) donde:
\begin{itemize}
\item G = (VN , VT , S, P ) es una gramática libre de contexto reducida y no ambigua.
\item A = $\cup_{X\in(VN \cup T)} A(X)$, es el conjunto finito de atributos (A(X) es el conjunto de atributos asociados al símbolo X)

\item V es el conjunto finito de dominios de valores de los atributos.
\item Dom : $A\rightarrow V$ asocia a cada atributo un dominio o conjunto de valores d ∈ V .
\item F es un conjunto finito de funciones semánticas de la forma:
\begin{equation}
f \subseteq (\bigotimes\limits_{j=0}^{k}{ Dom(a_{j} ))\rightarrow Dom(a_{0})}
\end{equation}

\item R = $\bigcup _{p∈P} R^{p}$ es el conjunto finito de reglas de atribución o ecuaciones asociadas a cada producción p ∈ P , donde
\begin{equation}
R^{p} = \bigcup\limits_{j=0}^{m^{p}}{\{r_{j}^{p}\}}\ \ \ \ \ \ (\#(R^{p} ) = m^{p} ≥ 0)
\end{equation}
y cada regla $r_{j}^{p} \in R^{p}$ , con 0 ≤ j ≤ $m^{p}$ es de la forma

\begin{equation}
r_{j}^{p}: X_{0}.a_{0} = f(X_{1}.a_{1} ,\dots , X_{k}.a_{k})
\end{equation} 
donde cada $X_{i}$ es un símbolo que ocurre en la producción \textit{p} , $a_{i} \in A(X_{i})$, ($0 \leqslant i \leqslant k$) y $f \in F$.

\end{itemize}
\end{definition}

Detalles de notación:
\begin{itemize}
\item Con respecto a CFG, en todos el desarrollo del trabajo se denotara a los símbolos no terminales con letras en mayúscula y a los símbolos terminales con letras en minúscula.
\item Se utilizará la notación \textbf{X.a} para significar que el atributo \texttt{a} está asociado al símbolo \texttt{X} (a $\in$ A(X)) y para denotar el valor de una ocurrencia o instancia del atributo \texttt{a} del símbolo \texttt{X} en una regla de atribución.

\end{itemize}
En \cite{tesismarcelo} (cap. 2) podemos ampliar con mayor detalle y formalismo lo analizado arriba.
\section{Preliminares}
\label{sec:pre-grafos}
\subsection{Grafo \textit{DP}}
Los grafos \textit{DP} denotan las relaciones de dependencias directas entre las instancias de la gramática. 

Un grafo \textit{DP} esta definido (\cite{estruc-algorit}) con las siguientes consideraciones: 

\begin{itemize}
\item Los nodos denotan \textit{instancias} de una producción.
\item Las aristas denotan la dependencia entre las instancias. Una arista $X_{i}.a\rightarrow X{j}.b$, \footnote{con \textit{i} y \textit{j} índices de ocurrencias consistentes con alguna ecuación de la gramática} denota que la evaluación de la instancia \textit{$X_{i}.a$} depende de la evaluación de \textit{$Y_{i}.b$} 
\end{itemize}

El conjunto de dependencias directas de una producción, de la gramática, se denota como \textit{DP(p)}(\textit{p} producción de la gramática) y se define como:
\begin{definition}
Dada una producción p de una gramática de atributos definida como en \ref{def:grammarattr}, entonces
\begin{equation}
DP(p) = \{(X_{i}.a, X_{j}.b) | X_{i}.a \rightarrow Y_{j}.b \in R^{p} \}
\end{equation}
\end{definition}

\subsection{Grafo \textit{Down}}
Los grafos \textit{Down} denotan las relaciones de los atributos de un símbolo. 

Un grafo \textit{Down} esta definido (\cite{estruc-algorit}) con las siguientes consideraciones: 

\begin{itemize}
\item Los nodos denotan \textit{atributos} de un símbolo.
\item Las aristas denotan la dependencia entre los atributos de un símbolo. Dado los atributos $a$ y $b\in A(X)$ del símbolo $X\in VN$, una arista $a\rightarrow b$, denota que la evaluación del atributo \texttt{a} depende de la evaluación de \texttt{b}. 
\end{itemize}
El conjunto de dependencias entre los atributos de un símbolo se denota como  
\texttt{Down(X)}($X\in VN$ de una gramática G) y se define:
\begin{definition}
Dada un símbolo X de una gramática de atributos definida como en \ref{def:grammarattr}, entonces
\begin{equation}
Down(X) = \{(a,b) | a \rightarrow b \} con\ a,b \in A(X)
\end{equation}
\end{definition}
\subsection{Grafo \textit{DCG}}

\textit{DCG} significa \textit{downward characteristic graphs}, los mismos contienen las dependencias entre instancias de la gramática para una producción \textit{p}, teniendo en cuenta un símbolo en la gramática.
\begin{definition}
Dado q una producción de la forma $X_{0}\rightarrow \alpha_{0} X_{1} \alpha_{1} X_{2} \dots X_{k} \alpha_{k}$, el \textit{downward characteristic graph} of $X_{0}$ en los subárboles derivados vía la producción \textit{q}, denotado como $DCG_{X_{0}}(q)$, es un grafo donde: 
\begin{itemize}
\item Los nodos son atributos del símbolo $X_{0}$.
\item Una arista, $X.a \rightarrow X.b$, denota una dependencia (transitiva) de X.b sobre X.a en algún subárbol derivado desde $X_{0}$ vía \textit{q}.
\end{itemize}
\end{definition}
Tomemos el siguiente teorema presentado por Wuu-Yang en \cite{wuu-yang1}:
\begin{theorem}
$\bigcup\limits_{\textit{todo p}}{DCG_{X} (p) = Down (X)}$
\end{theorem}
\underline{Nota:} $DCG_{X}(p)$ contiene las dependencias, entre las instancias de la gramática, para el símbolo \texttt{X}, acotando el análisis para la producción \textit{p} y los posibles contextos inferiores.

En la seccion \ref{XXX} se analiza el algoritmo para contruir los grafos DCG.

\subsection{Grafo \textit{ADP}}

Las siglas \textit{ADP} significan \textit{augmented dependency graph}. El grafo \textit{ADP} esta definido por instancias de la gramática, en los nodos, y cada arista se define como: $X_{i}.a\rightarrow X{j}.b$, denota que la evaluación de la instancia \textit{$X_{i}.a$} depende de la evaluación de \textit{$Y_{i}.b$}.

El conjunto de dependencias aumentadas se denota como $ADP (q | p_{1}, p_{2}, \dots, p_{k})$ y se define:
\begin{definition}
Sea q una producción de la forma $X_{0}\rightarrow \alpha_{0} X_{1} \alpha_{1} X_{2} \dots X_{k} \alpha_{k}$. Sea $p_{i}$ una producción cuya parte izquierda es $X_{i}$ ($1\leqslant i \leqslant k$). 
\begin{equation}
ADP (q | p_{1}, p_{2}, \dots, p_{k}) = DP(q) \bigcup\limits_{k}^{i=1}{DGC_{X_{i}}} (p_{i})
\end{equation}
\end{definition}

A partir de la definición anterior surge la siguiente:
\begin{definition}
El conjunto de todas las posibles dependencias aumentadas para una producción q se define como:
\begin{equation}
SADP(q) = \bigcup\limits_{q\in P}{ADP (q | p_{1}, p_{2}, \dots, p_{k})} 
\end{equation}
\end{definition}

\section{Definición MAG}

Una gramática \textit{G} de la forma \ref{def:grammarattr} es una \textit{gramática de atributos multi-planes} si y solo si 
\begin{equation}
\forall q : q \in P: (\forall g:g\ es\ un\ grafo\ de\ q \wedge g \in SADP(q) : g\ es\ no\ circular) 
\end{equation}

Las \textit{gramáticas multi-plans} mantienen propiedades deseables como:
\begin{itemize}
\item La clases de gramáticas MAG cumplen la propiedad de que todas las gramáticas de atributos en dicha clase son \textbf{no circulares}.
\item Toda gramática MAG cumple con las propiedades de gramática bien definida, esto es, todo árbol sintáctico derivado sobre una MAG contiene dependencias no circulares.
\end{itemize}

De los conceptos analizado arriba surgen los siguientes teoremas:

\begin{theorem}
$IDP-ANCAG(q) = \bigcup SADP(q)$, para toda producción q. 
\end{theorem}

\begin{theorem}
Toda gramática ANCAG es una gramática MAG, pero no viceversa.
\end{theorem}

