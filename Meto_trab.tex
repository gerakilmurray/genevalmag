\chapter{Metodología de trabajo}
\label{chap:metodologia}
\minitoc

\section{Prácticas de software}
El análisis y especificación de requerimientos puede parecer una tarea relativamente sencilla, pero la realidad es que el proceso de escribir un software requiere de un marco de trabajo para estructurar, planificar y controlar  el desarrollo del sistema.

Al mismo tiempo, el uso de herramientas en cada etapa del ciclo de vida (análisis, diseño, implementación y prueba), permite recorrer un camino de creación incremental del sistema, donde cada estadio del proceso refina el modelo. 

La importancia del uso de herramientas, modelos y métodos para asistir el proceso radica en visualizar y garantizar cualidades del producto desarrollado en practicas de software comprobadas teóricamente. 

Algunas de las prácticas de software se tratan en la siguiente sección:

\begin{description}
\item[\textbf{Análisis-Diseño}] Esta etapa se basó en el estudio del marco teórico compuesto por papers y libros propuestos por el director de tesis. Además, se concretaron reuniones frecuentes para evacuar dudas y tomar decisiones respecto a objetivos y aspectos a considerar en el modelo. Esta fase, también, se utilizó para familiarizarse y solidificar el manejo de herramientas empleadas en los distintos partes del proceso. 

\item[\textbf{Implementación}] En la etapa de implementación se invirtió una gran porción del tiempo total del proyecto. Esta fase, se dividió principalmente, en abordar las distintas etapas consideradas en el análisis-diseño, pero, también, el refinamiento del diseño era una tarea que jugaba un papel importante.

\item[\textbf{Prueba}] Esta etapa esta íntimamente relacionada con la etapa anterior (implementación), debido a que fueron realizadas en conjunto. Es decir, las pruebas eran abordadas luego de la implementación de cada fase distinguida en análisis-diseño. Para ello se planteaba casos de prueba específicos para cada fase, basándose en casos abordados en el marco teórico soporte del proyecto.

\item[\textbf{Documentación}] La documentación, a nivel de código, fue abordada desde la etapa de implementación hasta el fin del proceso. Esta etapa prioriza en hecho de clarificar detalles de implementación hacia la comunicación entre los desarrolladores, como así también para desarrolladores o posibles colaboradores externos al proyectos.

Además, en la parte final se utilizó full-time a la elaboración del informe, presentación y demás, que hacen al desarrollo de una tesina de carrera de grado.
\end{description}

\section{Lenguaje de programación C++}
C++ es un lenguaje de programación con tipado estático, multi-paradigma, compilado y de propósito general. Fue desarrollado por Bjarne Stroustrup\footnote{Científico de la computación, actualmente catedrático de ciencias de la computación en la Universidad A\&M de Texas. Página personal: \urllink{http://www2.research.att.com/~bs/homepage.html}} en el año 1979 en los ``Laboratorios Bell''\footnote{\urllink{http://en.wikipedia.org/wiki/Bell\_Labs}}, como una mejora al lenguaje de programación C, y fue originalmente llamado ``C con clases''.

El lenguaje ha evolucionado, ha sido estandarizado y aún continúa evolucionando. Actualmente, C++ soporta varios conceptos que permiten escribir programas con diferentes estilos: imperativo (procedural), orientado a objetos (herencia, polimorfismo, programación genérica, metaprogramación, etc).

La elección de C++ como lenguaje a utilizar en el desarrollo de \maggen\ surgió después de reuniones con el director de tesis, en las cuales de evaluaron lenguajes y se analizaron parámetros como lo son:

\begin{itemize}
\item Eficiencia en cuanto a tiempo de ejecución.

\item Posibilidad de redefinir de los operadores en un contexto dado, mediante la \textit{sobre carga de operadores}.

\item Utilización de librerías maduras como soporte de componentes necesarios y extras al objetivo de la tesis.

\item Entre otros.
\end{itemize}

Esta última, permitió la disponibilidad de bibliotecas genéricas, como la \textbf{STL} (\textit{Standard Template Library}) y \boost\ \textbf{\textit{Library C++}}(\cite{boost}) que fueron utilizadas para disponer de funcionalidades y estructuras con sólidas referencias.

Como bibliografía principal del lenguaje destacamos \cite{c++1} y \cite{c++2}.

\section{Herramientas}
\label{sec:metoherram}
La lista de herramientas que se detallan a continuación fueron utilizadas con resultados muy positivos en cada una de las etapas del desarrollo de sistema. Es de destacar que las herramientas son ``Free Software''.

\begin{description}
\item [Eclipse] es un entorno de desarrollo integrado de código abierto multiplataforma para desarrollar lo que el proyecto llama ``Aplicaciones de Cliente Enriquecido'', opuesto a las aplicaciones ``Cliente-liviano'' basadas en navegadores. Esta plataforma, típicamente ha sido usada para desarrollar entornos de desarrollo integrados (del inglés IDE). La versión utilizada fue ``\textit{Galileo}'', lanzada el 24 de junio del 2009. Más información, en página oficial: \urllink{http://www.eclipse.org}.

\item [Subversion] es un software de sistema de control de versiones. Es software libre bajo una licencia de tipo \textbf{\textit{Apache}/\textit{BSD}}\footnote{\urllink{http://en.wikipedia.org/wiki/Apache\_License} y \urllink{http://en.wikipedia.org/wiki/BSD\_licenses}} y se le conoce también como \textit{\textbf{svn}} por ser ese el nombre del comando que se utiliza. Esta herramienta fue de vital importancia para llevar a cabo la coordinación, comunicación y elaboración controlada entre los desarrolladores-autores del trabajo. Más información, en página oficial: \urllink{http://subversion.apache.org}.

\item [\LaTeXe] es una herramienta para sistema de composición de textos, orientado especialmente a la creación de libros, documentos científicos y técnicos que contengan fórmulas matemáticas. Este documento en su totalidad se escribió utilizando \LaTeXe\cite{latex}. Más información, en página oficial: \urllink{http://www.latex-project.org}.

\item [kile] es un editor \TeX/\LaTeX/\LaTeXe\ diseñado para el entorno de escritorios \textbf{KDE}\footnote{\urllink{http://www.kde.org/}}, permite editar documentos mediante comandos \LaTeX, además de visualizar, convertir o compilar documentos \LaTeX. Incorpora corrector ortográfico, resaltado de sintáxis, autocompletado, plantillas y patrones predefinidos y plegado de código, entre los más destacados y facilitando las tareas de edición. Más información, en página oficial: \urllink{http://kile.sourceforge.net}.

\item [Graphviz] es una aplicación de visualización de gráficos de código abierto que incluye un gran número de programas de trazado de gráficos. Además cuenta con interfaces interactivas y vía web, así como herramientas auxiliares y bibliotecas de funciones, existiendo versiones tanto para \textit{Windows} como para \textit{Linux}. Los gráficos son descriptos en archivos de texto plano con el lenguaje \textbf{DOT}\footnote{\urllink{http://en.wikipedia.org/wiki/DOT\_language}.} y pueden ser generados en múltiples formatos de salida como \textit{JPEG} (Joint Photographic Experts Group), \textit{PNG} (Portable Network Graphics) o \textit{SVG} (Scalable Vector Graphics). Dentro del paquete \textit{Graphviz} se encuentran distintos programas: \textbf{\texttt{dot}}, \textbf{\texttt{neato}}, \textbf{\texttt{fdp}} y \textbf{\texttt{twopi}}. Más información, en página oficial: \urllink{http://www.graphviz.org}.
Para la conversión de los archivos generados por \maggen, se utilizó el programa \texttt{\textbf{dot}}.

\item [Nemiver] es una interfaz gráfica para el depurador \texttt{\textbf{GNU gdb debugger}}\footnote{\urllink{http://www.gnu.org/software/gdb}}. Permite disponer de un entorno gráfico para depurar programas C/C++, visualizando el código fuente, el estado de las variables en memoria, entre otras funcionalidades. Más información, en página oficial: \urllink{http://projects.gnome.org/nemiver}.

\item [Dia] es un programa de creación de diagramas basado en \textbf{GTK+}\footnote{\urllink{http://www.gtk.org/}} bajo la licencia \textbf{GPL}. Puede ser usado para dibujar muchos tipos diferentes de diagramas. Dispone de una serie de extensiones para ayudar en la elaboración de diagramas entidad-interrelación, \textbf{UML}\footnote{\urllink{http://www.uml.org/}}, flujo de datos, diagramas de red, entre otros. Más información, en página oficial: \urllink{http://live.gnome.org/Dia}.

\item[Bouml] es un programa OpenSource y multiplataforma, prácticamente para cualquier distribución \textit{Unix}/\textit{Linux}/\textit{Solaris}, \textit{Mac} y \textit{Windows}. Además es realmente potente en cuanto a la generación automática de código a partir de los diagramas. Es una aplicación \textbf{UML2}\footnote{\urllink{http://www.uml.org/}} que permite definir y generar código en C++\footnote{\urllink{http://www2.research.att.com/~bs/C++.html}}, Java\footnote{\urllink{http://www.java.com/}}, IDL y PHP\footnote{\urllink{http://www.php.net/}}. Más información, en página oficial: \urllink{http://bouml.free.fr}.

\item[Cmake] es una herramienta que sirve para generar tanto proyectos y soluciones de \textit{Visual Studio}\footnote{\urllink{http://www.microsoft.com/visualstudio/en-us/products}}, como \texttt{\textbf{makefiles}} de \textit{Unix}. Permite gestionar un proyecto multiplataforma de forma muy sencilla, o centralizar la creación de nuevos proyectos de forma consistente. Otro conjunto de herramientas muy parecidas son las \textbf{\texttt{GNU build system}}\footnote{Comprende los programas de utilidad \textbf{\texttt{GNU Autoconf}}, \textbf{\texttt{Automake}} y \textbf{\texttt{Libtool}}.} conocidas también como \textbf{Autotools}. CMake es, a juicio de muchos, más sencillo de utilizar que las Autotools siendo, además, muy versátil, potente y escalable. Más información, en página oficial: \urllink{http://www.cmake.org/}.

\item [Análisis estático de código] entre los que se encuentran:

\begin{description}

\item [CCCC] \textbf{C} and \textbf{C}++ \textbf{C}ode \textbf{C}ounter. Fue desarrollado como un campo de pruebas para una serie de ideas relacionadas con las métricas de software en un proyecto de maestría. Más información, en página oficial: \urllink{http://cccc.sourceforge.net/}.

\item [\texttt{gcov}] es una herramienta GNU para realizar pruebas de cobertura sobre código fuente. Se utiliza en conjunto con \texttt{\textbf{gcc}}\footnote{\urllink{http://gcc.gnu.org}} para determinar el número de veces que cada línea de un programa se ejecuta durante una ejecución. Esto hace que sea posible encontrar áreas del código que no se utilizan, o que no se ejecutan en los casos de pruebas planteados. 

Dentro del desarrollo de \maggen, fue de gran ayuda un plugin para \textbf{Eclipse} que colorea las zonas no cubiertas\footnote{Para mayor detalle del plugin \urllink{http://sourceforge.jp/projects/ginkgo/wiki/EnglishPage}.}. Más información, en página oficial: \urllink{http://gcc.gnu.org/onlinedocs/gcc/Gcov.html}.

\end{description}

\end{description}